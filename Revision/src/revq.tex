\documentclass{beamer}
\usetheme{Madrid}
\usecolortheme{default}

\usepackage{minted}

\newcommand{\python}[1]{\mintinline{python}|#1|}

\title{COMP1117 Midterm Revision}
\author{Yuan Wenxuan}
\date{October 22, 2025}

\begin{document}

\frame{\titlepage}

\frame{
    \frametitle{Overview of Revision}
    \tableofcontents
}

\section{Recap of Key Concepts}
\subsection{Recap: Basic Syntax and Data Types}
\begin{frame}
    \frametitle{Recap: Data Types}
    \begin{itemize}
        \item Primitive Data Types
              \begin{itemize}
                  \item Integer (\python{int}): Whole numbers, e.g., \python{42}
                  \item Floating-point number (\python{float}): Decimal numbers, e.g., \python{3.14}
                  \item String (\python{str}): Text enclosed in quotes, e.g., \python{"Hello, World!"}
                  \item Boolean (\python{bool}): Represents truth values, either \python{True} or \python{False}
              \end{itemize}
        \item String Literals
              \begin{itemize}
                  \item Single quotes: \python{'Hello'}
                  \item Double quotes: \python{"Hello"}
                  \item Triple quotes (for multi-line strings):
                        \python{'''Hello'''} or \python{"""Hello"""}
                  \item Escape characters: \python{\n} (newline), \python{\t} (tab), \python{\\} (backslash), etc.
                  \item r-strings (raw strings): \python{r"C:\path\to\file"}
                  \item f-strings (formatted strings): \python{f"Hello, {name}!"}
              \end{itemize}
    \end{itemize}
\end{frame}
\begin{frame}
    \frametitle{Recap: Type Casting and Input}
    \begin{itemize}
        \item Type Casting
              \begin{itemize}
                  \item Convert one data type to another
                  \item Functions: \python{int()}, \python{float()}, \python{str()}, \python{bool()}
                  \item Example: \python{int("42")} converts the string "42" to the integer 42
              \end{itemize}
        \item Input
              \begin{itemize}
                  \item Get user input from the console
                  \item Function: \python{input()}
                  \item Note: Input is always treated as a string, so type casting may be needed
                  \item Example: \python{age = int(input("Enter your age: "))}
              \end{itemize}
    \end{itemize}
\end{frame}
\begin{frame}
    \frametitle{Recap: Variables and Naming Rules}
    \begin{itemize}
        \item Variables
              \begin{itemize}
                  \item Named storage for data
                  \item Created by assignment: \python{variable_name = value}
                  \item Example: \python{x = 10}, \python{name = "Alice"}
              \end{itemize}
        \item Naming Rules
              \begin{itemize}
                  \item Must start with a letter (a-z, A-Z) or an underscore (\_)
                  \item Can contain letters, digits (0-9), and underscores
                  \item Case-sensitive: \python{myVar} and \python{myvar} are different
                  \item Cannot be a reserved keyword (e.g., \python{if}, \python{for}, etc.)
              \end{itemize}
    \end{itemize}
\end{frame}

\subsection{Recap: Containers, Operators and Expressions}
\begin{frame}
    \frametitle{Recap: Containers and their Methods}
    \begin{itemize}
        \item \python{list}: mutable ordered collection
              \begin{itemize}
                  \item Methods: \python{list.append()}, \python{list.pop()}, \python{list.sort()}
                  \item Slicing: \python{list[start:end:step]}
                  \item List Comprehension: \python{[x * 2 for x in range(5)]}
              \end{itemize}
        \item \python{tuple}: immutable ordered collection
              \begin{itemize}
                  \item Tuple Packing and Unpacking \python{a, (b, c) = (1, (2, 4))}
              \end{itemize}
        \item \python{dict}: mutable collection of key-value pairs
              \begin{itemize}
                  \item Methods: \python{dict.keys()}, \python{dict.values()}, \python{dict.items()}
                  \item Accessing and Modifying values: \python{dict[key]}
              \end{itemize}
        \item \python{set}: mutable collection of unique elements
              \begin{itemize}
                  \item Methods: \python{set.add()}, \python{set.remove()}, \python{set.union()}
                  \item Access through index not supported
                  \item Check membership with \python{in}
              \end{itemize}
    \end{itemize}
\end{frame}
\begin{frame}
    \frametitle{Recap: Operators and Expressions}
    \begin{itemize}
        \item Statements vs Expressions
              \begin{itemize}
                  \item Statement: Performs an action (e.g., assignment, control flow)
                  \item Expression: Evaluates to a value (e.g., arithmetic operations)
                  \item In short: Statements do something, expressions are something
              \end{itemize}
        \item Arithmetic Operators
              \begin{itemize}
                  \item Basic math operations: \texttt{+}, \texttt{-}, \texttt{*}, \texttt{/}, \texttt{**}
                  \item Floor division: \texttt{//} (\textbf{rounds down} towards negative infinity)
                  \item Modulus: \texttt{\%} (the result holds the \textbf{same sign as the divisor})
              \end{itemize}
        \item Relational Operators
              \begin{itemize}
                  \item Compare values and get boolean: \texttt{==}, \texttt{!=}, \texttt{<}, \texttt{>}, \texttt{<=}, \texttt{>=}
                  \item Chained comparisons are supported in Python
                        (e.g., \texttt{a < b < c} equals \texttt{(a < b) and (b < c)})
              \end{itemize}
        \item Logical Operators
              \begin{itemize}
                  \item Combine boolean expressions: \texttt{and}, \texttt{or}, \texttt{not} (Short Circuited)
                  \item Precedence: \texttt{not} > \texttt{and} > \texttt{or} (Using parentheses is recommended)
              \end{itemize}
    \end{itemize}
\end{frame}

\subsection{Recap: Control Flow}
\begin{frame}[fragile]
    \frametitle{Recap: If-Elif-Else}
    \begin{itemize}
        \item Conditional Statements
        \item Different branches of code executed based on conditions
        \item Conditions will be evaluated to Boolean values
        \item Empty Containers, 0, None are considered \python{False}
              \begin{itemize}
                  \item Empty containers like: \python{[]} (list), \python{{}} (dict)
                  \item Empty string: \python{""} (string)
                  \item For Numeric Types: \python{0} (int), \python{0.0} (float)
                  \item Special None value: \python{None} (NoneType)
              \end{itemize}
        \item Other values are considered \python{True}
    \end{itemize}
\end{frame}
\begin{frame}[fragile]
    \frametitle{Recap: Loops and Control Flow}
    \begin{itemize}
        \item \python{for} loop: iterate over a sequence
        \item \python{while} loop: iterate while a condition is True
        \item \python{break}: exit the nearest enclosing loop
        \item \python{continue}: skip the current iteration and continue loop
        \item Nested Loops: loops inside loops
              \begin{itemize}
                  \item Inner loop completes all iterations for each iteration of outer loop
              \end{itemize}
        \item \python{range}: generate a sequence of numbers
              \begin{itemize}
                  \item \python{range(start)}: from 0 to (stop-1)
                  \item \python{range(start, stop)}: from start to (stop-1)
                  \item \python{range(start, stop, step)}: iterate with step (cannot be 0)
                  \item constraint: $(\texttt{stop} - \texttt{start}) \times \texttt{stop} > 0$, otherwise empty sequence
              \end{itemize}
        \item \python{enumerate(iterable)}: iterate pairs of (index, value)
    \end{itemize}
\end{frame}

\subsection{Recap: Functions}
\begin{frame}[fragile]
    \frametitle{Recap: Defining and Calling Functions}
    \begin{itemize}
        \item Defining: use \python{def} keyword to define a function
        \item Calling: function name followed by parentheses, e.g. \python{func(arg)}
        \item Parameters: variables listed when defining function
        \item \python{return}: use \python{return} keyword to return a value from the function
    \end{itemize}
    \begin{minted}{python}
    def function_name(arg1, arg2, ..., argN):
        ... # code block
        return value  # optional
        # None is returned if no return statement
    \end{minted}
\end{frame}
\begin{frame}
    \frametitle{Recap: Local, Global and Return}
    \begin{itemize}
        \item Local Variable: Defined inside a function
              \begin{itemize}
                  \item Accessible only within that function
                  \item Disappears after function finished execution
              \end{itemize}
        \item Global Variable: Defined outside any function
              \begin{itemize}
                  \item Accessible within or outside functions
                  \item Use \python{global} keyword to modify inside functions
                  \item Could be \textbf{shadowed} by local variables with the same name
              \end{itemize}
        \item Returning Values: Passed back to the caller using \python{return}
              \begin{itemize}
                  \item Multiple Return Values: use tuples to return multiple values
                        \python{return value1, value2, value3}
                  \item Useful combined with auto packing and unpacking
                        \python{value1, value2 = function_name()}
              \end{itemize}
    \end{itemize}
\end{frame}
\begin{frame}
    \frametitle{Recap: Default, Positional and Keyword Arguments}
    \begin{itemize}
        \item Positional Arguments: Passed based on their position
              \begin{itemize}
                  \item Order matters
                  \item Example: \python{func(10, 20)} passes 10 to first parameter, 20 to second
              \end{itemize}
        \item Keyword Arguments: Passed by explicitly naming parameters
              \begin{itemize}
                  \item Order does not matter
                  \item Example: \python{func(b=20, a=10)} passes 10 to \python{a}, 20 to \python{b}
              \end{itemize}
        \item Default Arguments: Parameters with default values
              \begin{itemize}
                  \item Can be omitted when calling the function
                  \item Example: \python{def func(a, b=5)} allows calling \python{func(10)} with \python{b} defaulting to 5
                  \item The default value will only be evaluated once at the definition
                  \item Use immutable types or \python{None} as default to avoid unexpected behavior
              \end{itemize}
    \end{itemize}
\end{frame}
\begin{frame}
    \frametitle{Recap: Lambda and Built-in Functions}
    \begin{itemize}
        \item Lambda Functions: Small anonymous functions defined with \python{lambda} keyword
              \begin{itemize}
                  \item Syntax: \python{lambda arguments: expression}
                  \item Can have any number of arguments but only one expression
                  \item Commonly used for short, throwaway functions
              \end{itemize}
        \item Built-in Functions: Predefined functions provided by Python
              \begin{itemize}
                  \item Examples: \python{len()}, \python{max()}, \python{min()}, \python{sum()}, \python{sorted()}
                  \item Used for common operations on data structures
              \end{itemize}
        \item Building Pipelines: Combine multiple functions for data processing
              \begin{itemize}
                  \item Example: \python{sorted(map(lambda x: x**2, nums))}
                  \item Applies \python{lambda} to each item in \python{nums}, then sorts the results
                  \item You may view Tutorial 4 Extension for more details
              \end{itemize}
    \end{itemize}
\end{frame}

\section{Practice: MCQs (33 Questions)}
\begin{frame}
    \frametitle{Practice: MCQs Overview}
    \begin{itemize}
        \item 33 Multiple Choice Questions (MCQs)
        \item Covering key concepts from previous sections
        \item Test understanding of Python basics and common pitfalls
        \item Solutions provided after each question
    \end{itemize}
\end{frame}

\begin{frame}[fragile]
    \frametitle{MCQ1}
    What is the output of the following code snippet?
    \begin{minted}{python}
    n = input("Please input an integer") # 12
    print(n * 2)
    \end{minted}

    \begin{itemize}
        \item A. \texttt{24}
        \item B. \texttt{1212}
        \item C. Error
        \item D. None of the above
    \end{itemize}
\end{frame}

\begin{frame}[fragile]
    \frametitle{MCQ2}
    What is the output of the following code snippet?
    \begin{minted}{python}
    if 0:
        print("a")
    elif -1:
        print("b")
    else:
        print("c")
    \end{minted}

    \begin{itemize}
        \item A. \texttt{a}
        \item B. \texttt{b}
        \item C. \texttt{c}
        \item D. Error message
    \end{itemize}
\end{frame}

\begin{frame}[fragile]
    \frametitle{MCQ3}
    What is the output of the following code snippet?
    \begin{minted}{python}
    s = ['l', 'o']
    x = 0
    for i in range(6):
        for j in range(6):
            print(s[x], end = '')
            x = (x + 1) % 2
    \end{minted}
    \begin{itemize}
        \item A. \texttt{lololo} (repeat 3 lines)
        \item B. \texttt{lolololololololololololololololololo}
        \item C. \texttt{olololololololololololololololololol}
        \item D. None of the above
    \end{itemize}
\end{frame}

\begin{frame}[fragile]
    \frametitle{MCQ4}
    What is the output of the following code snippet?
    \begin{minted}{python}
    a = 1
    b = 0
    c = 0
    if not a or b:
        print(1)
    elif not a or not b and c:
        print(2)
    elif not a or not b or not b and a:
        print(3)
    else:
        print(4)
    \end{minted}
    \begin{itemize}
        \item A. \texttt{1}
        \item B. \texttt{2}
        \item C. \texttt{3}
        \item D. \texttt{4}
    \end{itemize}
\end{frame}

\begin{frame}[fragile]
    \frametitle{MCQ5}
    What is the type of variable \texttt{a}?
    \begin{minted}{python}
    a = ("Name": "Peter", "UID": "12334", 
         "Age": "18", "Grade": "A")
    \end{minted}
    \begin{itemize}
        \item A. \python{list}
        \item B. \python{tuple}
        \item C. \python{dict}
        \item D. None of the above
    \end{itemize}
\end{frame}

\begin{frame}[fragile]
    \frametitle{MCQ6}
    What is the output of the following code snippet?
    \begin{minted}{python}
    print(9791286 % 10 // 3)
    \end{minted}
    \begin{itemize}
        \item A. \texttt{0}
        \item B. \texttt{1}
        \item C. \texttt{2}
        \item D. None of the above
    \end{itemize}
\end{frame}

\begin{frame}[fragile]
    \frametitle{MCQ7}
    What statement should be used to exit this while loop?
    \begin{minted}{python}
    while True:
        print("Hello")
    \end{minted}
    \begin{itemize}
        \item A. \texttt{break}
        \item B. \texttt{stop}
        \item C. \texttt{terminate}
        \item D. \texttt{continue}
    \end{itemize}
\end{frame}

\begin{frame}[fragile]
    \frametitle{MCQ8}
    What is the output of the following code snippet?
    \begin{minted}{python}
    s = "acbdefgh"
    m = ""
    for c in s:
        if c in "aeiou":
            continue
        m = m + c
    print(m)
    \end{minted}
    \begin{itemize}
        \item A. \texttt{acbdefgh}
        \item B. \texttt{ae}
        \item C. \texttt{cbdfgh}
        \item D. \texttt{fgh}
    \end{itemize}
\end{frame}

\begin{frame}[fragile]
    \frametitle{MCQ9}
    What is the output of the following code snippet?
    \begin{minted}{python}
    lc = [[[a, b]] for a in range(2) for b in range(3)]
    print(lc)
    \end{minted}
    \begin{itemize}
        \small
        \item A. \texttt{[[[1,1], [2,1]], [[1,2], [2,2]], [[1,3], [2,3]]]}
        \item B. \texttt{[[1, 1], [2, 1], [1, 2], [2, 2], [1, 3], [2, 3]]}
        \item C. \texttt{[[[0, 0], [1, 0]], [[0, 1], [1, 1]], [[0, 2], [1, 2]]]}
        \item D. Error
    \end{itemize}
\end{frame}

\begin{frame}[fragile]
    \frametitle{MCQ10}
    What is the output of the following code snippet?
    \begin{minted}{python}
    sc = set()
    for c in "ab,ac,bb,bc,":
        if c != 'a':
            sc.add(c)
    print(sc)
    \end{minted}
    \begin{itemize}
        \item A. \texttt{\{'b', 'c', 'bb', 'bc'\}}
        \item B. \texttt{\{'bb', 'bc'\}}
        \item C. \texttt{\{'bcbbbc'\}}
        \item D. \texttt{\{',', 'b', 'c'\}}
    \end{itemize}
\end{frame}

\begin{frame}[fragile]
    \frametitle{MCQ11}
    What is the output of the following code snippet?
    \begin{minted}{python}
    dc = dict()
    for i in range(1, 3):
        dc[i] = i * i * i
    print(dc[1], dc[2], dc[3])
    \end{minted}
    \begin{itemize}
        \item A. \texttt{1, 8, 27}
        \item B. \texttt{1, 2, 3}
        \item C. Error
        \item D. None of the above
    \end{itemize}
\end{frame}

\begin{frame}[fragile]
    \frametitle{MCQ12}
    How many lines of "Hello" will be printed by the following snippet?
    \begin{minted}{python}
    i = 0
    while i <= 2024:
        print("Hello")
        i = i + 1
    \end{minted}
    \begin{itemize}
        \item A. \texttt{2023}
        \item B. \texttt{2024}
        \item C. \texttt{2025}
        \item D. None of the above
    \end{itemize}
\end{frame}

\begin{frame}[fragile]
    \frametitle{MCQ13}
    What is the output of the following statement?
    \begin{minted}{python}
    print(int(int(int(30.4) / 3) / 2))
    \end{minted}
    \begin{itemize}
        \item A. \texttt{4}
        \item B. \texttt{5}
        \item C. \texttt{6}
        \item D. Error message
    \end{itemize}
\end{frame}

\begin{frame}[fragile]
    \frametitle{MCQ14}
    What is the output of the following statement?
    \begin{minted}{python}
    x = "\""
    y = "a\""
    print(x, y, sep = "\\")
    \end{minted}
    \begin{itemize}
        \item A. \texttt{\textbackslash"\textbackslash\textbackslash a\textbackslash"}
        \item B. \texttt{"\textbackslash\textbackslash a"}
        \item C. \texttt{"\textbackslash a"}
        \item D. None of the above
    \end{itemize}
\end{frame}

\begin{frame}[fragile]
    \frametitle{MCQ15}
    Which of the following variable names is invalid?
    \begin{itemize}
        \item A. \python{o0_0o}
        \item B. \python{python.py}
        \item C. \python{ILoveDonaldTrump}
        \item D. All of the above
    \end{itemize}
\end{frame}

\begin{frame}[fragile]
    \frametitle{MCQ16}
    What is the output of the following Python code snippet?
    \begin{minted}{python}
    T = ["happy", "sad", "boring"]
    T[-2] = "HKU"
    print(T)
    \end{minted}
    \begin{itemize}
        \item A. \texttt{["HKU", "sad", "boring"]}
        \item B. \texttt{["happy", "HKU", "boring"]}
        \item C. \texttt{["happy", "sad", "HKU"]}
        \item D. Error
    \end{itemize}
\end{frame}

\begin{frame}[fragile]
    \frametitle{MCQ17}
    What will be printed by the following Python code snippet?
    \begin{minted}{python}
    if 0:
        print("a", end="")
        print(1)
    elif "":
        print("b", end="")
        print(2)
    else:
        print("c", end="")
        print(3)
    \end{minted}
    \begin{itemize}
        \item A. \texttt{a1}
        \item B. \texttt{b2}
        \item C. \texttt{c3}
        \item D. Error
    \end{itemize}
\end{frame}

\begin{frame}[fragile]
    \frametitle{MCQ18}
    What is the value of the following expression?
    \begin{minted}{python}
    True or True and False
    \end{minted}
    \begin{itemize}
        \item A. \python{True}
        \item B. \python{False}
        \item C. Error
    \end{itemize}
\end{frame}

\begin{frame}[fragile]
    \frametitle{MCQ19}
    What is printed by the following Python code snippet?
    \begin{minted}{python}
    def f(n):
        (a, b) = (0, 1)
        while a < n:
            a, b = b, a+b
        return b
    print(int(f(8)))
    \end{minted}
    \begin{itemize}
        \item A. \texttt{5}
        \item B. \texttt{8}
        \item C. \texttt{13}
        \item D. \texttt{21}
    \end{itemize}
\end{frame}

\begin{frame}[fragile]
    \frametitle{MCQ20}
    What is the output of the following Python code snippet?
    \begin{minted}{python}
    print(3 == 2 + 1)
    \end{minted}
    \begin{itemize}
        \item A. Error
        \item B. \texttt{True}
        \item C. \texttt{1}
        \item D. \texttt{False}
    \end{itemize}
\end{frame}

\begin{frame}[fragile]
    \frametitle{MCQ21}
    What is the output of the following Python code snippet?
    \begin{minted}{python}
    x = 3
    if x > 2:
        x = x * 2
    if x > 4:
        x = 0
    print(x)
    \end{minted}
    \begin{itemize}
        \item A. \texttt{0}
        \item B. \texttt{3}
        \item C. \texttt{4}
        \item D. Error
    \end{itemize}
\end{frame}

\begin{frame}[fragile]
    \frametitle{MCQ22}
    What is the output of the following Python code snippet?
    \begin{minted}{python}
    x = "abcd"
    j = "a"
    while j in x:
        print(j, end='')
    \end{minted}
    \begin{itemize}
        \item A. no output
        \item B. print the letter \texttt{j} continuously, without stopping
        \item C. print the letter \texttt{a} continuously, without stopping
        \item D. Error
    \end{itemize}
\end{frame}

\begin{frame}[fragile]
    \frametitle{MCQ23}
    What is the output of the following program?
    \begin{minted}{python}
    x = -4
    a = [2, 4, 6, 8]
    for i in range(-2, 2):
        x = x + a[i]
    print(x)
    \end{minted}
    \begin{itemize}
        \item A. \texttt{8}
        \item B. \texttt{12}
        \item C. \texttt{16}
        \item D. Error
    \end{itemize}
\end{frame}

\begin{frame}[fragile]
    \frametitle{MCQ24}
    What is the output of the following program?
    \begin{minted}{python}
    i = 0
    for j in range(3):
        while True:
            if i == 3:
                break
            i = i + 1
        print("a", end='')
    print("b")
    \end{minted}
    \begin{itemize}
        \item A. \texttt{b}
        \item B. \texttt{aab}
        \item C. \texttt{aaab}
        \item D. Error
    \end{itemize}
\end{frame}

\begin{frame}[fragile]
    \frametitle{MCQ25}
    What is the output of the following Python code snippet?
    \begin{minted}{python}
    n = 8
    f = lambda x: x + n
    print(f(2))
    \end{minted}
    \begin{itemize}
        \item A. \texttt{9}
        \item B. \texttt{10}
        \item C. \texttt{11}
        \item D. Error
    \end{itemize}
\end{frame}

\begin{frame}[fragile]
    \frametitle{MCQ26}
    What is the output of the following snippet?
    \begin{minted}{python}
    def f(n):
        for _ in range(2, n//2+1):
            if n % _ == 0:
                return False
        return True

    print(list(filter(f, range(2, 11))))
    \end{minted}
    \begin{itemize}
        \item A. \texttt{[2, 3, 4, 5, 6, 7, 8, 9, 10]}
        \item B. \texttt{[True, True, False, True, False, True, False, False, False]}
        \item C. \texttt{[2, 3, 5, 7]}
        \item D. Error
    \end{itemize}
\end{frame}

\begin{frame}[fragile]
    \frametitle{MCQ27}
    What is the output of the following Python statement?
    \begin{minted}{python}
    print(list(map(int, ['2' * 3])))
    \end{minted}
    \begin{itemize}
        \item A. \texttt{[2]}
        \item B. \texttt{23}
        \item C. \texttt{[222]}
        \item D. Error
    \end{itemize}
\end{frame}

\begin{frame}[fragile]
    \frametitle{MCQ28}
    What is the output of the following program?
    \begin{minted}{python}
    c = 1
    x = 0
    while c <= 6:
        x = x + c
        c = c + 2
    print(x)
    \end{minted}
    \begin{itemize}
        \item A. \texttt{7}
        \item B. \texttt{8}
        \item C. \texttt{9}
        \item D. \texttt{12}
    \end{itemize}
\end{frame}

\begin{frame}[fragile]
    \frametitle{MCQ29}
    What is the output of the following program?
    \begin{minted}{python}
    def p(x):
        is_p = True
        for f in range(2, x):
            if x % f == 0:
                is_p = False
            if is_p == True:
                return "Yes"
            else:
                return "No"

    print(p(9))
    \end{minted}
    \begin{itemize}
        \item A. \texttt{Yes}
        \item B. \texttt{No}
        \item C. \texttt{True}
        \item D. Error
    \end{itemize}
\end{frame}

\begin{frame}[fragile]
    \frametitle{MCQ30}
    What is the output of the following program?
    \begin{minted}{python}
    def f(x, y, z):
        return max(x) + min(y * z)

    print(f([1, 2, 3], [4, 5, 6], 7))
    \end{minted}
    \begin{itemize}
        \item A. \texttt{7}
        \item B. \texttt{[3, 28]}
        \item C. \texttt{[3, 4]}
        \item D. Error
    \end{itemize}
\end{frame}

\begin{frame}[fragile]
    \frametitle{MCQ31}
    What is the output of the following snippet?
    \begin{minted}{python}
    alist = list(map(len, ["don't", "Hello",
        "is, are", "123"]))
    print(alist)
    \end{minted}
    \begin{itemize}
        \item A. \texttt{[4]}
        \item B. \texttt{[5, 5, 7, 3]}
        \item C. \texttt{[5, 5, 2, 3, 3]}
        \item D. Error
    \end{itemize}
\end{frame}

\begin{frame}[fragile]
    \frametitle{MCQ32}
    What is the output of the following snippet?
    \begin{minted}{python}
    d = {}
    d["a"] = [1, 2]
    d[2] = "Hello"
    d[(12, 3)] = "a"
    print(d[d[(12, 3)]])
    \end{minted}
    \begin{itemize}
        \item A. \texttt{"a"}
        \item B. \texttt{[1, 2]}
        \item C. \texttt{"Hello"}
        \item D. Error
    \end{itemize}
\end{frame}

\begin{frame}[fragile]
    \frametitle{MCQ33}
    What is the output of the following snippet?
    \begin{minted}{python}
    s = {1, 2, 2, 3, 3, 3, 4, 4, 4, 4}
    print(len(list(s)))
    \end{minted}
    \begin{itemize}
        \item A. \texttt{[1, 2, 3, 4]}
        \item B. \texttt{10}
        \item C. \texttt{4}
        \item D. None of the above
    \end{itemize}
\end{frame}

\end{document}
