\documentclass{beamer}
\usetheme{Madrid}
\usecolortheme{default}

\usepackage{minted}

\newcommand{\python}[1]{\mintinline{python}|#1|}

\title{COMP1117 Tutorial 9}
\author{YUAN Wenxuan}
\date{November 20, 2025}

\begin{document}

\frame{\titlepage}

\begin{frame}
    \frametitle{Overview of Today's Tutorial}
    \tableofcontents
\end{frame}

\section{Recap and Extensions}
\subsection{Recap: String Slicing}
\begin{frame}
    \frametitle{Recap: String Slicing}
    \begin{itemize}
        \item using colon \python{:} to slice strings
        \item syntax: \python{str[start:end:step]}
        \item default values: start = 0, end = len(str), step = 1
        \item negative indices and steps
        \item examples:
              \begin{itemize}
                  \item \python{"1234"[1:3] = "23"}
                  \item \python{"1234"[-3:-1] = "23"}
                  \item \python{"1234"[::-1] = "4321"}
              \end{itemize}
    \end{itemize}
\end{frame}
\subsection{Recap: String Methods}
\begin{frame}
    \frametitle{Recap: String Methods}
    \begin{itemize}
        \item Boolean methods: \python{isalpha()}, \python{isdigit()}, etc.
        \item Searching methods: \python{find()}, \python{index()}, \python{count()}
        \item Modification methods: \python{lower()}, \python{upper()}, \python{strip()}, \python{replace()}
        \item Splitting and joining: \python{split()}, \python{join()}
        \item Visit Python String Methods for more details at
              \href{https://docs.python.org/3/library/stdtypes.html}
              {https://docs.python.org/3/library/stdtypes.html}
    \end{itemize}
\end{frame}

\subsection{Recap: Higher-Order Functions and Lambdas}
\begin{frame}
    \frametitle{Recap: Higher-Order Functions and Lambdas}
    \begin{itemize}
        \item Functions as first-class citizens
        \item Higher-order functions: functions that take other functions as arguments or return functions as results
        \item Lambda: anonymous functions defined using the \python{lambda} keyword
        \item Common higher-order functions: \python{map}, \python{filter}, \python{reduce}
        \item Simple examples:
              \begin{itemize}
                  \item \python{map(int, input().split())} to convert input strings to integers
                  \item \python{filter(lambda x: x > 0, numbers)} to filter positive numbers
              \end{itemize}
        \item See previous Tutorial for more details
    \end{itemize}
\end{frame}

\subsection{Extension: Regex Basics}
\begin{frame}
    \frametitle{Extension: Regex Basics}
    \begin{itemize}
        \item Regex (Regular Expressions): a sequence of characters that define a search pattern
        \item Common regex symbols:
              \begin{itemize}
                  \item \texttt{.}: any character except newline
                  \item \texttt{*}: zero or more occurrences of the preceding element
                  \item \texttt{+}: one or more occurrences of the preceding element
                  \item \texttt{?}: zero or one occurrence of the preceding element
                  \item \texttt{\textasciicircum}: start of string
                  \item \texttt{\$}: end of string
                  \item \texttt{[abc]}: any character in the set (a, b, or c)
                  \item \texttt{[a-z]}: any character in the range (a to z)
                  \item \texttt{\textbackslash}: escape special characters
              \end{itemize}
        \item Example: \texttt{\textasciicircum a.*b\$} matches any string that starts with 'a' and ends with 'b'
    \end{itemize}
\end{frame}

\section{Practice Time!}
\begin{frame}
    \frametitle{Time to Practice!}
    \begin{itemize}
        \item MCQs
        \item Last word of input
        \item Max number in input
        \item Check credit card number
    \end{itemize}
\end{frame}

\subsection{MCQs}
\begin{frame}[fragile]
    \frametitle{Practice Time: MCQ1}
    What is the output of the following code?
    \begin{minted}{python}
s = "helloworld"
print(s[1:4], s[:5], s[4:], s[0:-1], s[:-1])
    \end{minted}
    \begin{itemize}
        \item A. \texttt{ell hello oworld helloworl helloworl}
        \item B. \texttt{ello hellow oworld helloworl lrowolleh}
        \item C. \texttt{hel hello world helloworl helloworl}
        \item D. \texttt{ell hello oworld helloworl lrowolleh}
    \end{itemize}
\end{frame}
\begin{frame}[fragile]
    \frametitle{Practice Time: MCQ1 Solution}
    What is the output of the following code?
    \begin{minted}{python}
s = "helloworld"
print(s[1:4], s[:5], s[4:], s[0:-1], s[:-1])
    \end{minted}
    \begin{itemize}
        \item \alert{A. \texttt{ell hello oworld helloworl helloworl}}
        \item B. \texttt{ello hellow oworld helloworl lrowolleh}
        \item C. \texttt{hel hello world helloworl helloworl}
        \item D. \texttt{ell hello oworld helloworl lrowolleh}
    \end{itemize}
\end{frame}

\begin{frame}[fragile]
    \frametitle{Practice Time: MCQ2}
    What is the output of the following code?
    \begin{minted}{python}
s1 = "Application"
s2 = s1.replace('a', 'A')
print(s2)
    \end{minted}
    \begin{itemize}
        \item A. \texttt{Application}
        \item B. \texttt{application}
        \item C. \texttt{ApplicAtion}
        \item D. \texttt{applicAtion}
    \end{itemize}
\end{frame}
\begin{frame}[fragile]
    \frametitle{Practice Time: MCQ2 Solution}
    What is the output of the following code?
    \begin{minted}{python}
s1 = "Application"
s2 = s1.replace('a', 'A')
print(s2)
    \end{minted}
    \begin{itemize}
        \item A. \texttt{Application}
        \item B. \texttt{application}
        \item \alert{C. \texttt{ApplicAtion}}
        \item D. \texttt{applicAtion}
    \end{itemize}
\end{frame}

\begin{frame}[fragile]
    \frametitle{Practice Time: MCQ3}
    What is the output of the following code?
    \begin{minted}{python}
s1 = "COMP,1117,Computer,Science"
s2 = ' '.join(s1.split(','))
print(s2)
    \end{minted}
    \begin{itemize}
        \item A. \texttt{COMP1117ComputerScience}
        \item B. \texttt{C O M P 1 1 1 7 C o m p u t e r S c i e n c e}
        \item C. \texttt{COMP,1117,Computer,Scince}
        \item D. \texttt{COMP 1117 Computer Science}
    \end{itemize}
\end{frame}
\begin{frame}[fragile]
    \frametitle{Practice Time: MCQ3 Solution}
    What is the output of the following code?
    \begin{minted}{python}
s1 = "COMP,1117,Computer,Science"
s2 = ' '.join(s1.split(','))
print(s2)
    \end{minted}
    \begin{itemize}
        \item A. \texttt{COMP1117ComputerScience}
        \item B. \texttt{C O M P 1 1 1 7 C o m p u t e r S c i e n c e}
        \item C. \texttt{COMP,1117,Computer,Science}
        \item \alert{D. \texttt{COMP 1117 Computer Science}}
    \end{itemize}
\end{frame}

\begin{frame}[fragile]
    \frametitle{Practice Time: MCQ4}
    What is the output of the following code?
    \begin{minted}{python}
def f1(x):
    return x < -1

L1 = [1, -2, -3, 4, 5]
m1 = list(map(f1, L1))
print(m1)
    \end{minted}
    \begin{itemize}
        \item A. Error
        \item B. \texttt{<map object at 0x...>}
        \item C. \texttt{[-2, -3]}
        \item D. \texttt{[False, True, True, False, False]}
    \end{itemize}
\end{frame}
\begin{frame}[fragile]
    \frametitle{Practice Time: MCQ4 Solution}
    What is the output of the following code?
    \begin{minted}{python}
def f1(x):
    return x < -1

L1 = [1, -2, -3, 4, 5]
m1 = list(map(f1, L1))
print(m1)
    \end{minted}
    \begin{itemize}
        \item A. Error
        \item B. \texttt{<map object at 0x...>}
        \item C. \texttt{[-2, -3]}
        \item \alert{D. \texttt{[False, True, True, False, False]}}
    \end{itemize}
\end{frame}

\begin{frame}[fragile]
    \frametitle{Practice Time: MCQ5}
    What is the output of the following code?
    \begin{minted}{python}
def f2(x):
    return x < 2

L2 = [1, -2, -3, 4, 5]
m2 = list(filter(f2, L2))
print(m2)
    \end{minted}
    \begin{itemize}
        \item A. Error
        \item B.\texttt{<filter object at 0x...>}
        \item C. \texttt{[1, -2, -3]}
        \item D. \texttt{[True, True, True, False, False]}
    \end{itemize}
\end{frame}
\begin{frame}[fragile]
    \frametitle{Practice Time: MCQ5 Solution}
    What is the output of the following code?
    \begin{minted}{python}
def f2(x):
    return x < 2

L2 = [1, -2, -3, 4, 5]
m2 = list(filter(f2, L2))
print(m2)
    \end{minted}
    \begin{itemize}
        \item A. Error
        \item B.\texttt{<filter object at 0x...>}
        \item \alert{C. \texttt{[1, -2, -3]}}
        \item D. \texttt{[True, True, True, False, False]}
    \end{itemize}
\end{frame}

\begin{frame}[fragile]
    \frametitle{Practice Time: MCQ6}
    What is the output of the following code?
    \begin{minted}{python}
L3 = [-2, 4]
m3 = map(lambda x: x*2, L3)
print(m3)
    \end{minted}
    \begin{itemize}
        \item A. Error
        \item B. \texttt{<map object at 0x...>}
        \item C. \texttt{[-4, 8]}
        \item D. None of the above
    \end{itemize}
\end{frame}
\begin{frame}[fragile]
    \frametitle{Practice Time: MCQ6 Solution}
    What is the output of the following code?
    \begin{minted}{python}
L3 = [-2, 4]
m3 = map(lambda x: x*2, L3)
print(m3)
    \end{minted}
    \begin{itemize}
        \item A. Error
        \item \alert{B. Address of m3 (a map object)}
        \item C. \texttt{[-4, 8]}
        \item D. None of the above
    \end{itemize}
\end{frame}

\subsection{Exercise: Last word of input}
\begin{frame}[fragile]
    \frametitle{Practice Time: Last word of input}

    Write a program that reads in a string $s$
    and then prints the last word in $s$.

    Note:

    The input string can contain letters, spaces and period symbols.
    The last word should only contain letters. You may need to remove
    the unnecessary symbol (e.g., period symbol) in the word.

    Hint:
    \begin{itemize}
        \item Use \python{str.split()} to obtain possible words
        \item Use negative index to get last word and last letter
    \end{itemize}

    Examples:
    \begin{center}
        \begin{tabular}{|c|c|}
            \hline
            Input                                 & Output           \\ \hline
            \texttt{Hello world.}                 & \texttt{world}   \\ \hline
            \texttt{This item sells for ten HKD.} & \texttt{HKD}     \\ \hline
            \texttt{This is an example}           & \texttt{example} \\ \hline
        \end{tabular}
    \end{center}
\end{frame}
\begin{frame}[fragile]
    \frametitle{Practice Time: Last word of input Solution}

    Solution using \python{filter()} and \python{str.isalpha()}:
    \begin{minted}{python}
word = input().split()[-1]
result = ''.join(filter(str.isalpha, word))
print(result)
    \end{minted}
    Alternative one-liner:
    \python{print(input().strip().split()[-1][-1])}

    Solution using \python{for} loop:
    \begin{minted}{python}
word = input().split()[-1]
result = ""
for char in word:
    if char.isalpha():
        result += char
print(result)
    \end{minted}
\end{frame}

\subsection{Exercise: Max number in input}
\begin{frame}[fragile]
    \frametitle{Practice Time: Max number in input}

    Write a program that read a line of digit-strings
    separated by comma and print the max number in them.

    Hints:
    \begin{itemize}
        \item Use \python{str.split()} to obtain number strings
        \item Convert the strings to integers
        \item Compare the values
    \end{itemize}

    Examples:
    \begin{center}
        \begin{tabular}{|c|c|}
            \hline
            Input                       & Output       \\ \hline
            \texttt{123,456,879,100}    & \texttt{879} \\ \hline
            \texttt{1,2,3,45,678,50,-4} & \texttt{678} \\ \hline
            \texttt{8,9,10,4,8}         & \texttt{10}  \\ \hline
        \end{tabular}
    \end{center}
\end{frame}
\begin{frame}[fragile]
    \frametitle{Practice Time: Max number in input Solution}

    Solution using \python{max()} and \python{map()}:
    \begin{minted}{python}
strings = input().split(',')
numbers = list(map(int, strings))
print(max(numbers))
    \end{minted}

    Solution using \python{for} loop:
    \begin{minted}{python}
strings = input().split(',')
result = int(strings[0])
for string in strings:
    number = int(string)
    if number > result:
        result = number
print(result)
    \end{minted}
\end{frame}

\subsection{Exercise: Check credit card number}
\begin{frame}[fragile]
    \frametitle{Practice Time: Check credit card number}

    Write a program that read a line of credit card number
    which each 4-digit parts are separated by space and
    use the Luhn Algorithm to check if the card number is valid.

    Luhn Algorithm is descripted as in the next frame.
    You need to use the Luhn Algorithm to check
    if the calculated check digit is the same as the one from the input.

    Examples:

    \begin{center}
        \begin{tabular}{|c|c|}
            \hline
            Input                        & Output           \\ \hline
            \texttt{5190 9902 8192 5290} & \texttt{Invalid} \\ \hline
            \texttt{6823 1198 3424 8189} & \texttt{Valid}   \\ \hline
            \texttt{3716 8200 1927 1998} & \texttt{Valid}   \\ \hline
        \end{tabular}
    \end{center}
\end{frame}

\begin{frame}[fragile]
    \frametitle{Practice Time: Check credit card number}

    Luhn Algorithm Procedures:
    \begin{enumerate}
        \small
        \item Remove the check digit (last digit) from the card number.
        \item Calculate the check digit
              \begin{enumerate}
                  \item Take the number and start from the rightmost digit
                        and move to left, double the value of every second digit.
                        (including the rightmost digit).
                  \item Calculate the sum of the digits of the result for each position.
                  \item Sum up the resulting values from all positions as $s$.
                  \item The calculated check digit is $10 - (s \mod 10)$.
              \end{enumerate}
        \item Compare your result with the check digit from input.
              The card number is valid if they are the same.
    \end{enumerate}
\end{frame}
\begin{frame}[fragile]
    \frametitle{Practice Time: Check credit card number Solution}

    Solution:
    \begin{minted}{python}
def calc_check_digit(digits: list[int]) -> int:
    result = 0
    for index in range(len(digits)):
        digit = digits[-index - 1]
        if index % 2 == 0:
            digit *= 2
        if digit > 9:
            digit -= 9
        result += digit  # at most 2 digits
    return (10 - (result % 10)) % 10
digits = list(map(int, input().replace(' ', '')))
if calc_check_digit(digits[:-1]) == digits[-1]:
    print("Valid")
else:
    print("Invalid")
    \end{minted}
\end{frame}

\section{Q\&A and Reminders}
\begin{frame}
    \frametitle{Any Questions?}

    Feel free to ask us about the tutorials and python coding!

    \begin{itemize}
        \item Contact us via email
        \item Ask us in person right now
        \item Post on Moodle or Ed
    \end{itemize}
\end{frame}

\begin{frame}
    \frametitle{Thank You!}

    Enjoy your coding journey in Python!

    \begin{itemize}
        \item Happy Coding!
        \item See you in the next tutorial (HW311, 10:00 - 11:50, Thursday)
        \item Submit your tutorial 9 exercises before \textbf{November 28 23:59}!
        \item Don't forget tutorial 7 and 8 exercises\\(due on \textbf{TOMORROW November 21 23:59})!
        \item Bonus: \LaTeX source code and compiled pdf of this slides available at
              \href{https://github.com/xtz206/COMP1117-Autumn25-Tutorials/releases}
              {https://github.com/xtz206/COMP1117-Autumn25-Tutorials/releases}
    \end{itemize}
\end{frame}

\end{document}