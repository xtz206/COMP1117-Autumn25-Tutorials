\documentclass{beamer}
\usetheme{Madrid}
\usecolortheme{default}

\usepackage{minted}
\usepackage{makecell}
\usepackage{hyperref}

\title{COMP1117 Tutorial 4}
\author{Yuan Wenxuan}
\date{October 9}

\begin{document}

\frame{\titlepage}

\begin{frame}
    \frametitle{Overview of Today's Tutorial}
    \tableofcontents
\end{frame}

\section{Recap and Extensions}
\subsection{Recap: Range and List Comprehension}

\begin{frame}
    \frametitle{Recap Lecture: Iterating Range}
    \begin{itemize}
        \item \mintinline{python}|range(stop)| (default \mintinline{python}|start=0| \mintinline{python}|step=1|)
        \item \mintinline{python}|range(start, stop)| (default \mintinline{python}|step=1|)
        \item \mintinline{python}|range(start, stop, step)|
        \item Constraints (Avoiding infinite sequence):
              \begin{itemize}
                  \item \mintinline{python}|step| cannot be zero (otherwise raises \mintinline{python}|ValueError|)
                  \item If \mintinline{python}|step| is positive, then \mintinline{python}|start < stop|
                  \item If \mintinline{python}|step| is negative, then \mintinline{python}|start > stop|
                  \item If constraints not satisfied, produces empty sequence
              \end{itemize}
    \end{itemize}
\end{frame}

\begin{frame}[fragile]
    \frametitle{Recap Lecture: List Comprehension}
    \begin{itemize}
        \item Concise way to create lists
        \item Syntax: \mintinline{python}|[expression for item in iterable <if condition>]|
        \item Example: Create a list of squares of even numbers from 0 to 9
              \begin{minted}{python}
squares_of_even = [x**2 for x in range(8) if x % 2 == 0]
print(squares_of_even)  # Output: [0, 4, 16, 36]
              \end{minted}
        \item Equivalent to:
              \begin{minted}{python}
squares_of_even = []
for x in range(10):
    if x % 2 == 0:
        squares_of_even.append(x**2)
              \end{minted}
    \end{itemize}
\end{frame}

\subsection{Extension: Other Comprehension (Optional)}

\begin{frame}[fragile]
    \frametitle{Extension: Other Comprehensions (Optional)}
    \begin{itemize}
        \item Generator Comprehension: Use parentheses instead of brackets
              \begin{minted}{python}
squares_gen = (x**2 for x in range(10) if x % 2 == 0)
print(next(squares_gen))  # Output: 0
print(next(squares_gen))  # Output: 4
              \end{minted}
        \item Set Comprehension: Use curly braces
              \begin{minted}{python}
squares_set = {x**2 for x in range(10) if x % 2 == 0}
print(squares_set)  # Output: {0, 64, 4, 36, 16}
              \end{minted}
        \item Dictionary Comprehension: Use curly braces with key-value pairs
              \begin{minted}{python}
squares_dict = {x: x**2 for x in range(3)}
print(squares_dict)  # Output: {0: 0, 1: 1, 2: 4}
              \end{minted}
    \end{itemize}
\end{frame}

\subsection{Extension: Iterables and Iterators (Optional)}
\begin{frame}
    \frametitle{Recap Tutorial: Iterable (Optional)}
    \begin{itemize}
        \item An object capable of returning its members one at a time
        \item Examples: list, tuple, dict, set, str, range, enumerate
        \item Can be used in a for-loop or with functions like \mintinline{python}|list()|, \mintinline{python}|tuple()|
        \item Can be collected into containers using:
              \begin{itemize}
                  \item \mintinline{python}|list(iterable)|
                  \item \mintinline{python}|tuple(iterable)|
                  \item \mintinline{python}|set(iterable)|
              \end{itemize}
    \end{itemize}
\end{frame}

\begin{frame}[fragile]
    \frametitle{Extension: Iterator (Optional)}
    \begin{itemize}
        \item An object representing a stream of data; returns one element at a time
        \item Created from an iterable using \mintinline{python}|iter()|
        \item Has a \mintinline{python}|__next__()| method to get the next item
        \item Raises \mintinline{python}|StopIteration| when no more items
        \item Example:
              \begin{minted}{python}
numbers = [1, 2, 3]
it = iter(numbers)
print(next(it))  # Output: 1
print(next(it))  # Output: 2
print(next(it))  # Output: 3
print(next(it))  # Raises StopIteration
              \end{minted}
        \item Custom Iterator: define \mintinline{python}|__iter__()| and \mintinline{python}|__next__()| methods
    \end{itemize}
\end{frame}

\begin{frame}[fragile]
    \frametitle{Extension: Why Iterator? (Optional)}
    \begin{itemize}
        \item Efficient memory usage: doesn't store all items in memory
        \item Can represent infinite sequences
        \item Example: Infinite sequence of natural numbers
              \begin{minted}{python}
from itertools import count
# Infinite iterator starting from 1
num_iter = count(1)
print(next(num_iter))  # Output: 1
print(next(num_iter))  # Output: 2
print(next(num_iter))  # Output: 3
              \end{minted}
        \item Compare with containers like \mintinline{python}|list|:
              \begin{minted}{python}
# Consumes memory for all numbers
num_list = list(range(1, 10**6))
              \end{minted}
    \end{itemize}
\end{frame}

\begin{frame}[fragile]
    \frametitle{Extension: Build a Pipeline (Optional)}
    \begin{itemize}
        \item iterable-returning functions
              \begin{itemize}
                  \item \mintinline{python}|map(func, iterable)|: applies \mintinline{python}|func| to each item
                  \item \mintinline{python}|filter(func, iterable)|: filters items where \mintinline{python}|func(item)| is \mintinline{python}|True|
                  \item \mintinline{python}|zip(*iterables)|: aggregates items from multiple iterables
              \end{itemize}
        \item Can be chained to form a processing pipeline
              \begin{minted}{python}
with open("data.txt") as file:
    for line in map(
        str.upper,  # convert to uppercase
        filter(
            lambda x: x.strip(),  # remove empty lines
            (l.strip() for l in file),  # remove '\n'
        ),
    ):
        print(line)
        # Prints non-empty lines in uppercase

              \end{minted}
    \end{itemize}
\end{frame}


\section{Practice Time!}

\begin{frame}
    \frametitle{Time to Practice!}
    \begin{itemize}
        \item MCQs
        \item Looping a Dictionary
        \item Count the digits
        \item Sum and Min
        \item Check perfect number
    \end{itemize}
\end{frame}


\subsection{MCQs}

\begin{frame}[fragile]
    \frametitle{Practice Time: MCQ1}
    What is the output of the following code?
    \begin{minted}{python}
    for i in range(5):
        print(i, end=',')
    \end{minted}
    \begin{itemize}
        \item A. \texttt{1,2,3,4,5,}
        \item B. \texttt{0,1,2,3,4,}
        \item C. \texttt{0,1,2,3,4,5,}
        \item D. \texttt{0 1 2 3 4}
    \end{itemize}
\end{frame}
\begin{frame}[fragile]
    \frametitle{Practice Time: MCQ1 Solution}
    What is the output of the following code?
    \begin{minted}{python}
    for i in range(5):
        print(i, end=',')
    \end{minted}
    \begin{itemize}
        \item A. \texttt{1,2,3,4,5,}
        \item \alert{B. \texttt{0,1,2,3,4,}}
        \item C. \texttt{0,1,2,3,4,5,}
        \item D. \texttt{0 1 2 3 4}
    \end{itemize}

    Explanation:

    \mintinline{python}|range(5)| yields \mintinline{python}|0..4|, and
    \mintinline{python}|print()| uses option \mintinline{python}|end=','|.
\end{frame}

\begin{frame}[fragile]
    \frametitle{Practice Time: MCQ2}
    What is the output of the following code?
    \begin{minted}{python}
    x = 'abcd'
    while j in x:
        print(j, end=' ')
    \end{minted}
    \begin{itemize}
        \item A. \texttt{abcd}
        \item B. \texttt{a b c d}
        \item C. \texttt{abcdabcdabcdabcd}
        \item D. Error
    \end{itemize}
\end{frame}
\begin{frame}[fragile]
    \frametitle{Practice Time: MCQ2 Solution}
    What is the output of the following code?
    \begin{minted}{python}
    x = 'abcd'
    while j in x:
        print(j, end=' ')
    \end{minted}
    \begin{itemize}
        \item A. \texttt{abcd}
        \item B. \texttt{a b c d}
        \item C. \texttt{abcdabcdabcdabcd}
        \item \alert{D. Error}
    \end{itemize}

    Explanation:

    variable \texttt{j} is not defined, leading to \mintinline{python}|NameError|.
\end{frame}

\begin{frame}[fragile]
    \frametitle{Practice Time: MCQ3}
    What is the output of the following code?
    \begin{minted}{python}
    x = 'abcd'
    j = 'j'
    while j in x:
        print(j, end=' ')
    \end{minted}
    \begin{itemize}
        \item A. no output
        \item B. \texttt{j j j j j...} (never stop)
        \item C. \texttt{a b c d}
        \item D. \texttt{abcd}
    \end{itemize}
\end{frame}
\begin{frame}[fragile]
    \frametitle{Practice Time: MCQ3 Solution}
    What is the output of the following code?
    \begin{minted}{python}
    x = 'abcd'
    j = 'j'
    while j in x:
        print(j, end=' ')
    \end{minted}
    \begin{itemize}
        \item \alert{A. no output}
        \item B. \texttt{j j j j j...} (never stop)
        \item C. \texttt{a b c d}
        \item D. \texttt{abcd}
    \end{itemize}

    Explanation:

    \mintinline{python}|'j'| not in \mintinline{python}|'abcd'|,
    while condition is \mintinline{python}{False} so loop body not executed.
\end{frame}

\begin{frame}[fragile]
    \frametitle{Practice Time: MCQ4}
    What is the output of the following code?
    \begin{minted}{python}
    x = 'abcd'
    j = 'a'
    while j in x:
        print(j, end=' ')
    \end{minted}
    \begin{itemize}
        \item A. no output
        \item B. \texttt{j j j j j ...} (never stop)
        \item C. \texttt{a a a a a ...} (never stop)
        \item D. \texttt{a}
    \end{itemize}
\end{frame}
\begin{frame}[fragile]
    \frametitle{Practice Time: MCQ4 Solution}
    What is the output of the following code?
    \begin{minted}{python}
    x = 'abcd'
    j = 'a'
    while j in x:
        print(j, end=' ')
    \end{minted}
    \begin{itemize}
        \item A. no output
        \item B. \texttt{j j j j j ...} (never stop)
        \item \alert{C. \texttt{a a a a a ...} (never stop)}
        \item D. \texttt{a}
    \end{itemize}

    Explanation:

    \mintinline{python}|j = 'a'| is in string \mintinline{python}|x = 'abcd'|
    and the while loop doesn't change \mintinline{python}|j| or \mintinline{python}|x|,
    so condition stays \mintinline{python}|True|.
\end{frame}

\begin{frame}[fragile]
    \frametitle{Practice Time: MCQ5}
    What is the output of the following code?
    \begin{minted}{python}
    i = 1
    while (i < 12):
        print(i, end=' ')
        i += 2
    \end{minted}
    \begin{itemize}
        \item A. \texttt{1}
        \item B. \texttt{2 4 6 8 10}
        \item C. \texttt{1 2 3 4 5 6}
        \item D. \texttt{1 3 5 7 9 11}
    \end{itemize}
\end{frame}
\begin{frame}[fragile]
    \frametitle{Practice Time: MCQ5 Solution}
    What is the output of the following code?
    \begin{minted}{python}
    i = 1
    while (i < 12):
        print(i, end=' ')
        i += 2
    \end{minted}
    \begin{itemize}
        \item A. \texttt{1}
        \item B. \texttt{2 4 6 8 10}
        \item C. \texttt{1 2 3 4 5 6}
        \item \alert{D. \texttt{1 3 5 7 9 11}}
    \end{itemize}

    Explanation:

    \mintinline{python}|i| starts at 1
    and increases by 2 while \mintinline{python}|i < 12|.
\end{frame}

\begin{frame}[fragile]
    \frametitle{Practice Time: MCQ6}
    What is the output of the following code?
    \begin{minted}{python}
    for i in range(0, 5, -1):
        print(i, end=' ')
    \end{minted}
    \begin{itemize}
        \item A. Error: invalid syntax
        \item B. Nothing will be printed
        \item C. \texttt{5 4 3 2 1}
        \item D. \texttt{0 -1 -2 -3 ...} (never stop)
    \end{itemize}
\end{frame}
\begin{frame}[fragile]
    \frametitle{Practice Time: MCQ6 Solution}
    What is the output of the following code?
    \begin{minted}{python}
    for i in range(0, 5, -1):
        print(i, end=' ')
    \end{minted}
    \begin{itemize}
        \item A. Error: invalid syntax
        \item \alert{B. Nothing will be printed}
        \item C. \texttt{5 4 3 2 1}
        \item D. \texttt{0 -1 -2 -3 ...} (never stop)
    \end{itemize}

    Explanation:

    \mintinline{python}|step| is -1 negative and
    \mintinline{python}|start| < \mintinline{python}|stop|,
    so the constraints are not satisfied,
    \mintinline{python}|range()| produces an empty sequence.
\end{frame}

\begin{frame}[fragile]
    \frametitle{Practice Time: MCQ7}
    What is the output of the following code?
    \begin{minted}{python}
    x = 2
    for i in range(x):
        x += 1
        print(x, end=' ')
    \end{minted}
    \begin{itemize}
        \item A. \texttt{0 1 2 3 4 ...} (never stop)
        \item B. \texttt{0 1 2 3}
        \item C. \texttt{0 1}
        \item D. \texttt{3 4}
    \end{itemize}
\end{frame}
\begin{frame}[fragile]
    \frametitle{Practice Time: MCQ7 Solution}
    What is the output of the following code?
    \begin{minted}{python}
    x = 2
    for i in range(x):
        x += 1
        print(x, end=' ')
    \end{minted}
    \begin{itemize}
        \item A. \texttt{0 1 2 3 4 ...} (never stop)
        \item B. \texttt{0 1 2 3}
        \item C. \texttt{0 1}
        \item \alert{D. \texttt{3 4}}
    \end{itemize}

    Explanation:

    \mintinline{python}|range(x)| is evaluated at loop start
    with \mintinline{python}|x=2| (\mintinline{python}|i=0,1|);
    \mintinline{python}|x| changes inside loop but iteration count fixed.
\end{frame}

\begin{frame}[fragile]
    \frametitle{Practice Time: MCQ8}
    What is the output of the following code?
    \begin{minted}{python}
    for i in range(1, 4):
        print(i, end=' ')
        i = i - 1
    \end{minted}
    \begin{itemize}
        \item A. \texttt{1 2 3}
        \item B. \texttt{1 0 -1}
        \item C. \texttt{1 0 -1 ...} (never stop)
        \item D. Nothing is printed
    \end{itemize}
\end{frame}
\begin{frame}[fragile]
    \frametitle{Practice Time: MCQ8 Solution}
    What is the output of the following code?
    \begin{minted}{python}
    for i in range(1, 4):
        print(i, end=' ')
        i = i - 1
    \end{minted}
    \begin{itemize}
        \item \alert{A. \texttt{1 2 3}}
        \item B. \texttt{1 0 -1}
        \item C. \texttt{1 0 -1 ...} (never stop)
        \item D. Nothing is printed
    \end{itemize}

    Explanation:

    for-loop gets next values from the iterator each iteration;
    assigning to loop variable doesn't affect iteration sequence.
\end{frame}

\begin{frame}[fragile]
    \frametitle{Practice Time: MCQ9}
    What is the output of the following code?
    \begin{minted}{python}
    x = 'abcd'
    for i in range(x):
        print(i, end=' ')
    \end{minted}
    \begin{itemize}
        \item A. \texttt{a b c d}
        \item B. \texttt{abcd}
        \item C. \texttt{abcdabcdabcdabcd}
        \item D. Error
    \end{itemize}
\end{frame}
\begin{frame}[fragile]
    \frametitle{Practice Time: MCQ9 Solution}
    What is the output of the following code?
    \begin{minted}{python}
    x = 'abcd'
    for i in range(x):
        print(i, end=' ')
    \end{minted}
    \begin{itemize}
        \item A. \texttt{a b c d}
        \item B. \texttt{abcd}
        \item C. \texttt{abcdabcdabcdabcd}
        \item \alert{D. Error}
    \end{itemize}

    Explanation:

    \mintinline{python}|range()| expects integer arguments;
    passing a string raises \mintinline{python}|TypeError|.
\end{frame}

\subsection{Exercise: Looping a Dictionary}
\begin{frame}[fragile]
    \frametitle{Practice Time: Looping a Dictionary}
    Write a program to store the following dictionary:
    \begin{minted}{python}
    card = {
        'a':1, 'b':2, 'c':3, '4':4, '5':5, '6':6, '7':7,
        '8':8, '9':9, '10':10, 'o':11, 'p':12, 'q':13
    }
    \end{minted}
    Using a for loop to print all the key-value pairs in this dictionary.

    \textbf{Note}: The symbol between key and value is the minus sign \mintinline{python}|'-'|.

    Expected Output:
    \begin{minted}{text}
    a-1
    b-2
    c-3
    4-4
    5-5
    ...
    q-13
    \end{minted}
\end{frame}
\begin{frame}[fragile]
    \frametitle{Practice Time: Looping a Dictionary Solution}
    Solution:
    \begin{minted}{python}
card = {
    'a':1, 'b':2, 'c':3, '4':4, '5':5, '6':6, '7':7,
    '8':8, '9':9, '10':10, 'o':11, 'p':12, 'q':13
}
for key in card:
    print(f"{key}-{card[key]}")
    \end{minted}

    (Optional) Alternative using \mintinline{python}|dict.items()| and \mintinline{python}|sep='-'|:
    \begin{minted}{python}
for key, value in card.items():
    print(key, value, sep='-')
    \end{minted}

    (Optional) One-Liner:
    \begin{minted}{python}
print(*(f"{key}-{card[key]}" for key in card), sep='\n')
    \end{minted}
\end{frame}

\subsection{Exercise: Count the digits}
\begin{frame}[fragile]
    \frametitle{Practice Time: Count the digits}

    Write a program to read a string entered by the user and
    then use a for loop to count the number of digits in it.

    \textbf{Hints}: The value of the numeric characters is
    greater or equal to the character \mintinline{python}|'0'|
    and smaller or equal to the character \mintinline{python}|'9'|.

    Examples:

    \begin{tabular}{|c|c|}
        \hline
        Input                                   & Output     \\ \hline
        \texttt{Hello 123 world!}               & \texttt{3} \\ \hline
        \texttt{This item sells for 55610 HKD.} & \texttt{5} \\ \hline
    \end{tabular}
\end{frame}
\begin{frame}[fragile]
    \frametitle{Practice Time: Count the digits Solution}

    Solution:
    \begin{minted}{python}
    count = 0
    for char in input():
        if '0' <= char <= '9':
            count += 1
    print(count)
    \end{minted}

    (Optional) Alternative using \mintinline{python}|str.isdigit()|:
    \begin{minted}{python}
    count = 0
    for char in input():
        if char.isdigit():
            count += 1
    print(count)
    \end{minted}

    (Optional) One-Liner:
    \begin{minted}{python}
    print(sum(1 for c in input() if c.isdigit()))
    \end{minted}
\end{frame}

\subsection{Exercise: Sum and Min}
\begin{frame}[fragile]
    \frametitle{Practice Time: Sum and Min}

    Write a program to use a while loop to read the numbers
    from the user repeatedly until he/she inputs 0.
    Then report the sum of these numbers and the minimum value among them.

    \textbf{Note}: You don't need to store all the numbers

    Examples:

    \begin{tabular}{|c|c|}
        \hline
        Input & Output                          \\ \hline
        \makecell[tl]{\texttt{5}                \\ \texttt{8} \\ \texttt{4} \\ \texttt{-5} \\ \texttt{0}}
              & \makecell[tl]{\texttt{sum = 12} \\ \texttt{min = -5}} \\ \hline
        \makecell[tl]{\texttt{3}                \\ \texttt{15} \\ \texttt{0}}
              & \makecell[tl]{\texttt{sum = 18} \\ \texttt{min = 3}} \\ \hline
    \end{tabular}
\end{frame}
\begin{frame}[fragile]
    \frametitle{Practice Time: Sum and Min Solution}

    Solution:
    \begin{minted}{python}
    n = int(input())
    s = 0 # sum
    m = n # min
    while n != 0:
        s += n
        if n < m:
            m = n
    print(f"sum={s}")
    print(f"min={m}")
    \end{minted}

    (Optional) Alternative using syntactic sugars (Python 3.8+):
    \begin{minted}{python}
    s = 0, m = float("inf")
    while (n := int(input())) != 0:
        s += n
        m = n if n < m else m
    print(f"sum={s}\nmin={m}")
    \end{minted}
\end{frame}

\subsection{Exercise: Check perfect number}
\begin{frame}[fragile]
    \frametitle{Practice Time: Check perfect number}
    Write a program to read a positive integer $N$ ($N \ge 2$),
    test if $N$ is perfect number and output the answer with either
    \mintinline{python}|True| or \mintinline{python}|False|.
    Then the user decides whether to check another number or not.
    If the user input \mintinline{python}|'Y'|,
    then let the user to input another integer, otherwise stop.

    \textbf{Note}: A perfect number is a number that equals
    the sum of all its factors excluding itself.
    For example, 28 is a perfect number because the sum of
    its factors excluding itself $1+2+4+7+14 = 28$.

    Definition: \href{https://en.wikipedia.org/wiki/Perfect_number}{https://en.wikipedia.org/wiki/Perfect\_number}

    Examples:

    \begin{tabular}{|c|c|}
        \hline
        Input & Output               \\ \hline
        \makecell[tl]{\texttt{28}    \\ \texttt{N}} & \makecell[tl]{\texttt{True}} \\ \hline
        \makecell[tl]{\texttt{4}     \\ \texttt{Y} \\ \texttt{5} \\ \texttt{N}} &
        \makecell[tl]{\texttt{False} \\ \texttt{False}} \\ \hline
    \end{tabular}
\end{frame}
\begin{frame}[fragile]
    \frametitle{Practice Time: Check perfect number Solution}

    Solution:
    \begin{minted}{python}
    while True:
        n = int(input())
        s = 0
        for i in range(1, n):
            if n % i == 0:
                s += i
        print(s == N)
        if input() != 'Y':
            break
    \end{minted}

    (Optional) Alternative using generator comprehension and \mintinline{python}|sum()|:
    \begin{minted}{python}
    while True:
        n = int(input())
        # One-Liner to check perfect number
        print(sum(i for i in range(1, n) if n % i == 0) == n)
        if input() != 'Y':
            break
    \end{minted}
\end{frame}

\section{Q\&A and Conclusion}
\begin{frame}
    \frametitle{Any Questions?}

    Feel free to ask us about the tutorials and python coding!

    \begin{itemize}
        \item Contact us via email
        \item Ask us in person right now
        \item Post on Moodle or Ed
    \end{itemize}
\end{frame}

\begin{frame}
    \frametitle{Thank You!}

    Enjoy your coding journey in Python!

    \begin{itemize}
        \item Happy Coding!
        \item See you in the next tutorial (HW311, 10:00 - 11:50, Thursday)
        \item Submit your tutorial 4 exercises before \textbf{October 17 23:59}!
        \item Don't forget tutorial 3 exercises\\
              (due on \textbf{TOMORROW October 10 23:59})!
        \item Bonus: \LaTeX source code and compiled pdf of this slides available at
              \href{https://github.com/xtz206/COMP1117-Autumn25-Tutorials/releases}
              {https://github.com/xtz206/COMP1117-Autumn25-Tutorials/releases}
    \end{itemize}
\end{frame}

\end{document}
