\documentclass{beamer}
\usetheme{Madrid}
\usecolortheme{default}

\usepackage{minted}

\newcommand{\python}[1]{\mintinline{python}|#1|}

\title{COMP1117 Tutorial 6}
\author{YUAN Wenxuan}
\date{October 30, 2025}

\begin{document}

\frame{\titlepage}

\begin{frame}
    \frametitle{Overview of Today's Tutorial}
    Hope you did well in the quiz!
    \tableofcontents
\end{frame}

\section{Recap and Extensions}
\subsection{Recap: Recursion}
\begin{frame}[fragile]
    \frametitle{Recap: Recursion}
    \begin{itemize}
        \item Recursion: A function calling itself, while having a base case to stop
        \item Key components:
              \begin{itemize}
                  \item Base Case (When to stop)
                  \item Recursive Case (The problem size reduces each time)
              \end{itemize}
        \item Example: Factorial Function
              \begin{minted}{python}
def factorial(n):
    if n == 0:
        return 1
    return n * factorial(n - 1)
        \end{minted}
    \end{itemize}
\end{frame}
\begin{frame}
    \frametitle{Recap: Recursion (Cont.)}
    \begin{itemize}
        \item Recursion can simplify code for problems like tree traversal, combinatorial problems, etc.
        \item Examples of recursive problems:
              \begin{itemize}
                  \item Fibonacci Sequence
                  \item Tower of Hanoi
                  \item Permutations and Combinations
              \end{itemize}
        \item Common pitfalls:
              \begin{itemize}
                  \item Missing base case
                  \item Not reducing problem size
                  \item Stack overflow for deep recursion
              \end{itemize}
    \end{itemize}
\end{frame}
\begin{frame}[fragile]
    \frametitle{Recap: Quick Exponentiation vs Linear Exponentiation}
    \begin{minted}{python}
    def qexp(base, exp):
        if exp == 0 or base == 1:
            return 1
        if base == 0:
            return 0
        if exp % 2 == 1:
            return base * qexp(base, exp - 1)
        half = qexp(base, exp // 2)
        return half * half
    def lexp(base, exp):
        result = 1
        while exp > 0:
            exp -= 1
            result *= base
        return result
              \end{minted}
\end{frame}

\subsection{Extension: Introduction to Python OOP}
\begin{frame}[fragile]
    \frametitle{Extension: Classes and Objects}
    \begin{itemize}
        \item Everything in Python is an object
        \item Class: Blueprint for creating objects
        \item Object: Instance of a class
    \end{itemize}

    Example:
    \begin{minted}[fontsize=\small]{python}
    class Car:
        def __init__(self, speed: int, color: str):
            self.speed = speed
            self.color = color
        def accelerate(self, increment: int) -> None:
            self.speed += increment        
        def __str__(self) -> str:
            return f"{self.color} car moving at {self.speed} km/h" 
    my_car = Car(10, "red")
    print(my_car.speed)  # Output: 10
    my_car.accelerate(40)
    print(my_car)  # Output: red car moving at 50 km/h
    \end{minted}
\end{frame}

\begin{frame}
    \frametitle{Extension: Attributes and Methods}
    \begin{itemize}
        \item Attributes: Variables that belong to an object
        \item Methods: Functions that belong to an object
        \item \python{self}: Refers to the current instance of the class
        \item OOP Principles:
              \begin{itemize}
                  \item Encapsulation
                  \item Inheritance
                  \item Polymorphism
              \end{itemize}
        \item Improve Code Organization and Reusability
    \end{itemize}
\end{frame}

\begin{frame}
    \frametitle{Extension: Magic Methods}
    \begin{itemize}
        \item Special methods with double underscores (e.g., \python{__init__}, \python{__str__})
        \item Allow customization of object behavior
        \item Common magic methods:
              \begin{itemize}
                  \item \python{__init__}: Constructor
                  \item \python{__str__}: String representation
                  \item \python{__add__}: Addition operator
                  \item \python{__len__}: Length of object
                  \item \python{__eq__}: Equality comparison
                  \item \python{__hash__}: Hash value (required for dictionary keys)
              \end{itemize}
    \end{itemize}
\end{frame}

\begin{frame}[fragile]
    \frametitle{Extension: Quick Demo of Complex Number Class}
    \begin{minted}[fontsize=\small]{python}
    class Complex:
        def __init__(self, re: float, im: float):
            self.re = re
            self.im = im
        def __add__(self, other):
            return Complex(self.re + other.re,
                           self.im + other.im)
        def __str__(self):
            if self.im < 0:
                return f"{self.re} - {-self.im}i"
            return f"{self.re} + {self.im}i"
    c1 = Complex(2, 3)
    c2 = Complex(4, 5)
    c3 = c1 + c2
    print(c3)  # Output: 6 + 8i
    \end{minted}
\end{frame}

\subsection{Extension: Type Hint}
\begin{frame}[fragile]
    \frametitle{Extension: Type Hint}
    \begin{itemize}
        \item Type hints: Optional annotations to indicate expected data types
        \item Improve code readability and help with static analysis
        \item Syntax:
              \begin{itemize}
                  \item Function parameters: \python{def func(param: T) -> R:}
                  \item Variables: \python{var: T = value}
                  \item Containers: \python{lst: list[T] = []}
              \end{itemize}
    \end{itemize}
    Example:
    \begin{minted}{python}
    def greet(name: str) -> str:
        return f"Hello, {name}!"
    age: int = 25
    numbers: list[int] = [1, 2, 3]
    \end{minted}
\end{frame}

\section{Practice Time!}
\begin{frame}
    \frametitle{Time to Practice!}
    \begin{itemize}
        \item Even Product Calculator
        \item Multiply Two Number
        \item Special Sequence
        \item Digital Calculator
    \end{itemize}
\end{frame}


\subsection{Exercise: Even Product Calculator}
\begin{frame}[fragile]
    \frametitle{Practice Time: Even Product Calculator}

    Implement a recursive function to calculate the product of
    all the positive even numbers not larger than n (n>=2).
    For example, if n = 7, the return value of this function will be 48.
    That’s because:
    \begin{enumerate}
        \item All the positive even number no larger than n = 7 is 6, 4, and 2
        \item Their product is 6 * 4 * 2 = 48
    \end{enumerate}
    Given Code:
    \begin{minted}{python}
    def product(n):
        # Your implementation here
    print(product(int(input())))
    \end{minted}
    Examples:

    \begin{center}
        \small
        \begin{tabular}{|c|c|}
            \hline
            Input & Output     \\ \hline
            3     & 2          \\ \hline
            5     & 8          \\ \hline
            10    & 3840       \\ \hline
            20    & 3715891200 \\ \hline
            15    & 645120     \\ \hline
        \end{tabular}
    \end{center}
\end{frame}

\begin{frame}[fragile]
    \frametitle{Practice Time: Even Product Calculator Solution}

    Solution:
    \begin{minted}{python}
    def product(n):
        if n == 2:
            return 2
        if n % 2 == 0:
            return n * product(n - 2)
        return product(n - 1)
    \end{minted}
\end{frame}

\subsection{Exercise: Multiply Two Number}
\begin{frame}[fragile]
    \frametitle{Practice Time: Multiply Two Number}

    Implement a recursive function \python{multiply(a, b)} to
    calculate the value of \python{a * b} without using the \python{*} operator
    \begin{itemize}
        \item $a, b \ge 0$ and $\min(a, b) \le 1000$
        \item Hint: $a * b = a + a(b - 1)$ or $b + b(a - 1)$
    \end{itemize}
    Given Code:
    \begin{minted}{python}
    def multiply(a, b):
        # Your implementation here
    a = int(input())
    b = int(input())
    print(multiply(a, b))
    \end{minted}

    Examples (New Line Separated Inputs):
    \begin{center}
        \small
        \begin{tabular}{|c|c|}
            \hline
            Input  & Output \\ \hline
            0 0    & 0      \\ \hline
            50 100 & 5000   \\ \hline
            1 1000 & 1000   \\ \hline
            1000 1 & 1000   \\ \hline
        \end{tabular}
    \end{center}
\end{frame}
\begin{frame}[fragile]
    \frametitle{Practice Time: Multiply Two Number Solution}

    Solution:
    \begin{minted}{python}
    def multiply(a, b):
        if a == 0 or b == 0:
            return 0
        if a < b:
            return b + multiply(a - 1, b)
        return a + multiply(a, b - 1)
    \end{minted}
\end{frame}

\subsection{Exercise: Special Sequence}
\begin{frame}[fragile]
    \frametitle{Practice Time: Special Sequence}

    There is a number sequence:
    $2, 1, 3, 2, 3, 6, 6, 18, 36, 108, 648, 3888, 69984 \cdots$

    Try to implement the \python{func(n)} to calculate
    the $n$-th value of this number sequence. ($n \le 1$)
    The first three numbers are 2, 1, and 3.
    The fourth number is the product of the first two numbers.
    The the pattern continues like this.

    Given Code:
    \begin{minted}{python}
    def func(n):
        # Your implementation here
    n = int(input())
    print(func(n))
    \end{minted}

    Examples:

    \begin{center}
        \small
        \begin{tabular}{|c|c|}
            \hline
            Input & Output                              \\ \hline
            1     & 2                                   \\ \hline
            20    & 32889049376007652207159000669618176 \\ \hline
            15    & 2720977920                          \\ \hline
        \end{tabular}
    \end{center}
\end{frame}

\begin{frame}[fragile]
    \frametitle{Practice Time: Special Sequence Solution}

    Solution:
    \begin{minted}{python}
    def func(n):
        if n == 1:
            return 2
        if n == 2:
            return 1
        if n == 3:
            return 3
        return func(n - 2) * func(n - 3)
    \end{minted}
\end{frame}

\subsection{Exercise: Digital Calculator}
\begin{frame}[fragile]
    \frametitle{Practice Time: Digital Calculator}

    Try to implement func(n, digit) to calculate the times
    that the digit ($0 \le \text{digit} \le 9$) shown
    in the interval $[1, n]$.

    For instance, if $n = 12$, $\text{digit} = 2$, the output will be 2.
    Because in $[1, 12]$, all the numbers are 1, 2, 3, 4, 5, 6, 7, 8, 9, 10, 11, 12.
    There are two numbers containing the digit 2, which are 2, and 12.
    So the return value in this case is 2.

    Another example is that if $n = 11$, $\text{digit} = 1$, the output will be 4.
    Because in $[1, 11]$, all the numbers are $1, 2, 3, 4, 5, 6, 7, 8, 9, 10, 11$.
    There are 3 numbers containing the digit 1, which are 1, 10, and 11.
    Furthermore, there are two $1$ in the number 11.
    Therefore, the return value in this case is $1 + 1 + 2 = 4$.
\end{frame}

\begin{frame}[fragile]
    \frametitle{Practice Time: Digital Calculator (Cont.)}
    Given Code:
    \begin{minted}{python}
    def func(n, digit):
        # Your implementation here
    n = int(input())
    digit = int(input())
    print(func(n, digit))
    \end{minted}

    Examples (New Line Separated Inputs):

    \begin{center}
        \begin{tabular}{|c|c|}
            \hline
            Input & Output \\ \hline
            11 1  & 4      \\ \hline
            22 2  & 6      \\ \hline
            20 1  & 12     \\ \hline
        \end{tabular}
    \end{center}
\end{frame}

\begin{frame}[fragile]
    \frametitle{Practice Time: Digital Calculator Solution}

    Solution using \python{str.count()}:
    \begin{minted}{python}
    def func(n, digit):
        if n == 0:
            return 0
        return str(n).count(str(digit)) + \
            func(n - 1, digit)
    \end{minted}

    Solution using only integer operations:
    \begin{minted}{python}
    def func(n, digit):
        if n == 0:
            return 0
        retval = func(n - 1, digit)
        while n > 0:
            if n % 10 == digit:
                retval += 1
            n //= 10
        return retval
    \end{minted}
\end{frame}

\section{Q\&A and Reminders}
\begin{frame}
    \frametitle{Any Questions?}

    Feel free to ask us about the tutorials and python coding!

    \begin{itemize}
        \item Contact us via email
        \item Ask us in person right now
        \item Post on Moodle or Ed
    \end{itemize}
\end{frame}

\begin{frame}
    \frametitle{Thank You!}

    Enjoy your coding journey in Python!

    \begin{itemize}
        \item Happy Coding!
        \item See you in the next tutorial (HW311, 10:00 - 11:50, Thursday)
        \item Submit your tutorial 6 exercises before \textbf{November 7 23:59}!
        \item Don't forget tutorial 5 exercises\\(due on \textbf{TOMORROW October 31 23:59})!
        \item Bonus: \LaTeX source code and compiled pdf of this slides available at
              \href{https://github.com/xtz206/COMP1117-Autumn25-Tutorials/releases}
              {https://github.com/xtz206/COMP1117-Autumn25-Tutorials/releases}
    \end{itemize}
\end{frame}

\end{document}