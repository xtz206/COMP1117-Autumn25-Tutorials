\documentclass{beamer}
\usetheme{Madrid}
\usecolortheme{default}

\usepackage{minted}

\title{COMP1117 Tutorial 1}
\author{Yuan Wenxuan}
% \institute{HKU}
\date{September 18, 2025}

\begin{document}

\frame{\titlepage}

\begin{frame}
    \frametitle{About Us (Student TAs)}

    \begin{itemize}
        \item Yuan Wenxuan (Peter)
              \begin{itemize}
                  \item Year 2 Computer Science Student (BEng(CompSc))
                  \item Contact me: \href{mailto:u3629274@connect.hku.hk}{u3629274@connect.hku.hk}
              \end{itemize}
        \item Yip Wing Long (Cyrus)
              \begin{itemize}
                  \item Contact: \href{mailto:cwlyip@connect.hku.hk}{cwlyip@connect.hku.hk }
              \end{itemize}
    \end{itemize}
\end{frame}

\begin{frame}
    \frametitle{Please Note...}

    \begin{itemize}
        \item Our First Tutorial Today
              \begin{itemize}
                  \item Online via Zoom
                  \item Try to install Python interpreter on your computer
              \end{itemize}
        \item Later Tutorials
              \begin{itemize}
                  \item \textbf{Thursday 10:00 - 11:50}, \textbf{HW311}, Face to Face
                  \item No attendance taken (But recommended to attend)
              \end{itemize}
        \item Contents
              \begin{itemize}
                  \item Recap the lectures
                  \item Go through the slides provided by the teaching team
                  \item Solve the tutorial exercises
                  \item Q\&A: Deal with your confusions
              \end{itemize}
        \item \textbf{Tutorial Exercises} (10\% of your final grade!)
              \begin{itemize}
                  \item Deadline: \textbf{next Friday 23:59} of the tutorial class
                  \item Example: For Sept 18 T1, the deadline is \textbf{Sept 26 23:59}
                  \item Late work is \textbf{NOT} accepted
              \end{itemize}
    \end{itemize}
\end{frame}

\begin{frame}
    \frametitle{Recapping the Lectures: Overview}
    In previous lectures, we have learned:
    \begin{itemize}
        \item Data Types: \mintinline{python}|int|, \mintinline{python}|float|, \mintinline{python}|str|, \mintinline{python}|bool|
        \item Type Casting and Input
        \item Variables and Naming Rules
        \item Sequences and Containers: \texttt{list}, \texttt{tuple}, \texttt{dict}, \texttt{set}
        \item Operators and Expressions (Arithmetic, Comparison, Logical)
    \end{itemize}
\end{frame}

\begin{frame}
    \frametitle{Recapping the Lectures: Data Types}
    \begin{itemize}
        \item Primitive Data Types
              \begin{itemize}
                  \item Integer (\mintinline{python}|int|): Whole numbers, e.g., \mintinline{python}|42|
                  \item Floating-point number (\mintinline{python}|float|): Decimal numbers, e.g., \mintinline{python}|3.14|
                  \item String (\mintinline{python}|str|): Text enclosed in quotes, e.g., \mintinline{python}|"Hello, World!"|
                  \item Boolean (\mintinline{python}|bool|): Represents truth values, either \mintinline{python}|True| or \mintinline{python}|False|
              \end{itemize}
        \item String Laterals
              \begin{itemize}
                  \item Single quotes: \mintinline{python}|'Hello'|
                  \item Double quotes: \mintinline{python}|"Hello"|
                  \item Triple quotes (for multi-line strings):
                        \mintinline{python}|'''Hello'''| or \mintinline{python}|"""Hello"""|
                  \item Escape characters: \mintinline{python}|\n| (newline), \mintinline{python}|\t| (tab), \mintinline{python}|\\| (backslash), etc.
                  \item r-strings (raw strings): \mintinline{python}|r"C:\path\to\file"|
                  \item f-strings (formatted strings): \mintinline{python}|f"Hello, {name}!"|
              \end{itemize}
    \end{itemize}
\end{frame}

\begin{frame}
    \frametitle{Recapping the Lectures: Type Casting and Input}
    \begin{itemize}
        \item Type Casting
              \begin{itemize}
                  \item Convert one data type to another
                  \item Functions: \mintinline{python}|int()|, \mintinline{python}|float()|, \mintinline{python}|str()|, \mintinline{python}|bool()|
                  \item Example: \mintinline{python}|int("42")| converts the string "42" to the integer 42
              \end{itemize}
        \item Input
              \begin{itemize}
                  \item Get user input from the console
                  \item Function: \mintinline{python}|input()|
                  \item Note: Input is always treated as a string, so type casting may be needed
                  \item Example: \mintinline{python}|age = int(input("Enter your age: "))|
              \end{itemize}
    \end{itemize}
\end{frame}

\begin{frame}
    \frametitle{Recapping the Lectures: Variables and Naming Rules}
    \begin{itemize}
        \item Variables
              \begin{itemize}
                  \item Named storage for data
                  \item Created by assignment: \mintinline{python}|variable_name = value|
                  \item Example: \mintinline{python}|x = 10|, \mintinline{python}|name = "Alice"|
              \end{itemize}
        \item Naming Rules
              \begin{itemize}
                  \item Must start with a letter (a-z, A-Z) or an underscore (\_)
                  \item Can contain letters, digits (0-9), and underscores
                  \item Case-sensitive: \mintinline{python}|myVar| and \mintinline{python}|myvar| are different
                  \item Cannot be a reserved keyword (e.g., \mintinline{python}|if|, \mintinline{python}|for|, etc.)
              \end{itemize}
        \item Naming Better (Optional)
              \begin{itemize}
                  \item Use meaningful names which reflect its purpose
                  \item Use snake\_case for multi-word variable names: \mintinline{python}|my_variable_name|
                  \item  snake\_case is more recommended in Python
                  \item Use all uppercase letters for constants: \mintinline{python}|CONST|
              \end{itemize}
    \end{itemize}
\end{frame}

\begin{frame}
    \frametitle{Recapping the Lectures: Sequences and Containers}
    \begin{itemize}
        \item List (\texttt{list})
              \begin{itemize}
                  \item Ordered, mutable collection of items
                  \item Created using square brackets: \mintinline{python}|my_list = [1, 2, 3]|
                  \item Access elements using indices: \mintinline{python}|my_list[0]| gives 1
              \end{itemize}
        \item Tuple (\texttt{tuple})
              \begin{itemize}
                  \item Ordered, immutable collection of items
                  \item Created using parentheses: \mintinline{python}|my_tuple = (1, 2, 3)|
                  \item Single-element tuple: \mintinline{python}|single_tuple = (1,)| (note the comma)
                  \item Access elements using indices: \mintinline{python}|my_tuple[0]| gives 1
              \end{itemize}
        \item Mutable vs Immutable
              \begin{itemize}
                  \item Mutable: Can be changed after creation (e.g., \texttt{list})
                  \item Immutable: Cannot be changed after creation (e.g., \texttt{tuple})
              \end{itemize}
    \end{itemize}
\end{frame}

\begin{frame}
    \frametitle{Recapping the Lectures: Sequences and Containers}
    \begin{itemize}
        \item Dictionary (\texttt{dict})
              \begin{itemize}
                  \item Unordered collection of key-value pairs
                  \item Created using curly braces: \mintinline{python}|my_dict = {"key": "value"}|
                  \item Access values using keys: \mintinline{python}|my_dict["key"]| gives "value"
              \end{itemize}
        \item Set (\texttt{set})
              \begin{itemize}
                  \item Unordered collection of unique items
                  \item Created using curly braces: \mintinline{python}|my_set = {1, 2, 3}|
                  \item No indexing, but can check membership: \mintinline{python}|1 in my_set| gives \mintinline{python}|True|
              \end{itemize}
        \item Note: Both are mutable
    \end{itemize}
\end{frame}


\begin{frame}
    \frametitle{Recapping the Lectures: Operators and Expressions}

    \begin{itemize}
        \item Statements vs Expressions
              \begin{itemize}
                  \item Statement: Performs an action (e.g., assignment, control flow)
                  \item Expression: Evaluates to a value (e.g., arithmetic operations)
                  \item In short: Statements do something, expressions are something
              \end{itemize}
        \item Arithmetic Operators
              \begin{itemize}
                  \item Basic math operations: \texttt{+}, \texttt{-}, \texttt{*}, \texttt{/}, \texttt{**}
                  \item Floor division: \texttt{//} (\textbf{rounds down} towards negative infinity)
                  \item Modulus: \texttt{\%} (the result holds the \textbf{same sign as the divisor})
              \end{itemize}
        \item Relational Operators
              \begin{itemize}
                  \item Compare values and get boolean: \texttt{==}, \texttt{!=}, \texttt{<}, \texttt{>}, \texttt{<=}, \texttt{>=}
                  \item Chained comparisons are supported in Python (e.g., \texttt{a < b < c} equals \texttt{(a < b) and (b < c)})
              \end{itemize}
        \item Logical Operators
              \begin{itemize}
                  \item Combine boolean expressions: \texttt{and}, \texttt{or}, \texttt{not} (Short Circuited)
                  \item Precedence: \texttt{not} > \texttt{and} > \texttt{or} (Using parentheses is recommended)
              \end{itemize}
    \end{itemize}

\end{frame}


\begin{frame}
    \frametitle{Make it Run!}
    Before coding, you need to know how to run your Python code!
    \begin{itemize}
        \item \textbf{VPL on Moodle}
              \begin{itemize}
                  \item Important!
                  \item How you submit your tutorial exercises
              \end{itemize}
        \item Interactive Mode in IDLE
        \item Script Mode in IDLE
        \item IDEs (e.g. PyCharm, VSCode, etc.)
        \item Command Line / Terminal
    \end{itemize}
\end{frame}

\begin{frame}
    \frametitle{VPL on Moodle: Most Important!}
    \begin{itemize}
        \item VPL = Virtual Programming Lab
        \item use \texttt{Edit} page to edit, run and evaluate your code (recommended)
        \item upload your file in the \texttt{Submission} page
        \item Demonstration of using VPL
    \end{itemize}
\end{frame}

\begin{frame}
    \frametitle{IDLE: Interactive and Script Mode}
    \begin{itemize}
        \item IDLE = Python's Integrated Development and Learning Environment
        \item Easy to use and often comes pre-installed with Python Interpreter
        \item Two modes:
              \begin{itemize}
                  \item Interactive Mode (Useful for testing small code snippets)
                  \item Script Mode (Useful for writing longer programs)
              \end{itemize}
        \item Demonstration of using IDLE
    \end{itemize}
\end{frame}

\begin{frame}
    \frametitle{Time to Practice!}

    Let's go through the tutorial exercises now!
    \begin{itemize}
        \item Hello World
        \item Printing Patterns
        \item Calculate the Area of a Triangle
        \item Time Conversion
        \item Repeating a Pattern
        \item MCQs
    \end{itemize}
\end{frame}

\begin{frame}
    \frametitle{Any Questions?}

    Feel free to ask us about the tutorials and python coding!

    \begin{itemize}
        \item Contact us via email
        \item Ask in the chat box or in-person (in later f2f tutorials)
        \item Post on the course forum on Moodle or Ed
    \end{itemize}
\end{frame}

\begin{frame}
    \frametitle{Thank You!}

    Wish you all a great semester in COMP1117!

    \begin{itemize}
        \item Happy Coding!
        \item See you in the next tutorial (HW311, 10:00 - 11:50, Thursday)
        \item Submit your Tutorial 1 exercises by \textbf{September 26, 23:59}!
        \item Bonus: \LaTeX \  Source code of the slide available at
              \href{https://github.com/xtz206/COMP1117-Autumn25-Tutorials}{https://github.com/xtz206/COMP1117-Autumn25-Tutorials}
    \end{itemize}
\end{frame}

\end{document}