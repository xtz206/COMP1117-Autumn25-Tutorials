\documentclass{beamer}
\usetheme{Madrid}
\usecolortheme{default}

\usepackage{minted}
\usepackage{makecell}

\newcommand{\python}[1]{\mintinline{python}|#1|}

\title{COMP1117 Tutorial 7}
\author{YUAN Wenxuan}
\date{November 6, 2025}

\begin{document}

\frame{\titlepage}

\begin{frame}
    \frametitle{Overview of Today's Tutorial}
    \tableofcontents
\end{frame}

\section{Recap and Extensions}
\subsection{Recap: Functions and Functional Programming}
\begin{frame}
    \frametitle{Recap: Functions}
    \begin{itemize}
        \item Defining functions using \python{def}
        \item Returning values using \python{return}
        \item Function parameters and arguments
        \item Scope of variables in functions
        \item See Tutorial 5 for more details
    \end{itemize}
\end{frame}
\begin{frame}
    \frametitle{Recap: Higher-Order Functions and Lambdas}
    \begin{itemize}
        \item Functions as first-class citizens
        \item Higher-order functions: functions that take other functions as arguments or return functions as results
        \item Lambda: anonymous functions defined using the \python{lambda} keyword
        \item Common higher-order functions: \python{map}, \python{filter}, \python{reduce}
        \item Simple examples:
              \begin{itemize}
                  \item \python{map(int, input().split())} to convert input strings to integers
                  \item \python{filter(lambda x: x > 0, numbers)} to filter positive numbers
              \end{itemize}
        \item See Tutorial 5 for more details
    \end{itemize}
\end{frame}
% \subsection{Recap: TODO3RC}
% \begin{frame}
%     \frametitle{Recap: TODO3RC}
%     \begin{itemize}
%         % TODO: Recap items
%         \item
%     \end{itemize}
% \end{frame}

\subsection{Extension: Introduction to OOP (Cont.)}
\begin{frame}[fragile]
    \frametitle{Extension: Inheritance and Polymorphism}
    \begin{itemize}
        \item Inheritance allows a class (subclass) to inherit attributes and methods from another class (superclass)
        \item Syntax: \python{class Subclass(Superclass):}
        \item Polymorphism allows methods to do different things based on the object it is acting upon
        \item Example: (See source code for details)
    \end{itemize}
\end{frame}
\subsection{Extension: Exceptions and Error Handling}
\begin{frame}[fragile]
    \frametitle{Extension: Exceptions and Error Handling}
    \begin{itemize}
        \item Exceptions are errors that occur during the execution of a program
        \item Common exceptions: \python{ZeroDivisionError}, \python{ValueError}, \python{TypeError}, etc.
        \item Use \python{try} and \python{except} blocks to handle exceptions
        \item Example:
              \begin{minted}{python}
try:
    result = 10 / int(input("Enter a number: "))
    print("Result:", result)
except ZeroDivisionError:
    print("Error: Division by zero is not allowed.")
except ValueError:
    print("Error: Please enter a valid integer.")
              \end{minted}
    \end{itemize}
\end{frame}

\section{Practice Time!}
\begin{frame}
    \frametitle{Time to Practice!}
    \begin{itemize}
        \item One-Line UNIX Command
        \item Multiple Commands using map
        \item Checking Command using filter
    \end{itemize}
\end{frame}

\subsection{Exercise: One-Line UNIX Command}
\begin{frame}[fragile]
    \frametitle{Practice Time: One-Line UNIX Command}

    Write a Python program that implements an interactive file system shell.
    The program should accept a command from the user and processing it, printing the result.

    Supported Commands:
    \begin{itemize}
        \item \texttt{LS}: List content of the current directory.
        \item \texttt{MKDIR <dirname>}: Create a new directory (must start with \python{"/"})
        \item \texttt{TOUCH <filename>}: Create a new empty file
        \item \texttt{FILTER <pattern>}: Filter files/directories by suffix or prefix
    \end{itemize}

    Initial File System State:

    The file system starts with three empty directories in the root:
    \texttt{/home}, \texttt{/Desktop}, and \texttt{/Documents}
    and one file \texttt{test.txt}.
\end{frame}
\begin{frame}[fragile]
    \frametitle{Practice Time: One-Line UNIX Command (Cont.)}

    Program Behaviors:
    \begin{itemize}
        \item Print the output the command immediately
        \item Expect that all input is correct
        \item All list outputs should be sorted alphanumerically
        \item For \texttt{FILTER} command,
              \begin{itemize}
                  \item For files: filter by suffix (e.g., \texttt{.txt})
                  \item For directories: filter by prefix \python{"/"} (only directories start with \python{"/"})
                  \item Returns matching files or directories
              \end{itemize}
    \end{itemize}

    Examples:
    \begin{center}
        \small
        \begin{tabular}{|l|l|}
            \hline
            Input                    & Output                                                   \\ \hline
            \texttt{LS}              & \python{['/Desktop', '/Documents', '/home', 'test.txt']} \\ \hline
            \texttt{MKDIR /projects} & \python{Directory '/projects' created}                   \\ \hline
            \texttt{TOUCH readme.md} & \python{File 'readme.md' created}                        \\ \hline
            \texttt{FILTER .txt}     & \python{['test.txt']}                                    \\ \hline
            \texttt{FILTER /}        & \python{['/Desktop', '/Documents', '/home']}             \\ \hline
            \texttt{FILTER .md}      & \python{[]}                                              \\ \hline
        \end{tabular}
    \end{center}
\end{frame}
\begin{frame}[fragile]
    \frametitle{Practice Time: One-Line UNIX Command Solution}

    Solution:
    \begin{minted}{python}
files = ["/home", "/Desktop", "/Documents", "test.txt"]
def main():
    tokens = input().strip().split()
    command = tokens[0]
    args = tokens[1:]

    if command == "LS":
        print(ls())
    elif command == "MKDIR":
        print(mkdir(args[0]))
    elif command == "TOUCH":
        print(touch(args[0]))
    elif command == "FILTER":
        print(filter_items(args[0]))
    \end{minted}
\end{frame}
\begin{frame}[fragile]
    \frametitle{Practice Time: One-Line UNIX Command Solution (Cont.)}

    Solution (Cont.):
    \begin{minted}{python}
def ls():
    return sorted(files)
def mkdir(dirname):
    if dirname not in files:
        files.append(dirname)
    return f"Directory '{dirname}' created"
def touch(filename):
    if filename not in files:
        files.append(filename)
    return f"File '{filename}' created"
def filter_items(pattern):
    if pattern == "/":
        return sorted(f for f in files if f.startswith("/"))
    return sorted(f for f in files if f.endswith(pattern) \
        and not f.startswith("/"))
        \end{minted}
\end{frame}

\subsection{Exercise: Multiple Commands}
\begin{frame}[fragile]
    \frametitle{Practice Time: Multiple Commands}

    Expand your solution to process multiple file system commands supplied
    as a single input string with commands separated by semicolons.
    Use the map function to apply your file system operations to each command sequentially.

    Input Format:
    A single string containing multiple commands separated by double ampersands (\texttt{\&\&}):

    \texttt{"COMMAND1 arg1 \&\& COMMAND2 arg2 \&\& COMMAND3 arg3"}

    Example 1:
    \begin{itemize}
        \small
        \item Input: \texttt{MKDIR /work \&\& TOUCH main.py \&\& LS}
        \item Output:
              \begin{minted}{python}
[
    "Directory '/work' created", "File 'main.py' created",
    [
        "/Desktop", "/Documents", "/home", 
        "/work", "main.py", "test.txt"
    ]
]
                \end{minted}
    \end{itemize}
\end{frame}
\begin{frame}[fragile]
    \frametitle{Practice Time: Multiple Commands (Cont.)}
    Example 2:
    \begin{itemize}
        \small
        \item Input: \texttt{TOUCH file1.txt \&\& TOUCH file2.py \&\& FILTER .txt}
        \item Output:
              \begin{minted}{python}
[
    "File 'file1.txt' created", "File 'file2.py' created",
    ["file1.txt", "test.txt"]
]
              \end{minted}
    \end{itemize}

    Example 3:
    \begin{itemize}
        \small
        \item Input: \texttt{MKDIR /data \&\& FILTER /}
        \item Output:
              \begin{minted}{python}
[
    "Directory '/data' created",
    ["/Desktop", "/Documents", "/data", "/home"]
]
              \end{minted}
    \end{itemize}
\end{frame}

\begin{frame}[fragile]
    \frametitle{Practice Time: Multiple Commands Solution}

    Solution:
    \begin{minted}{python}
def processCommand(line):
    tokens = line.strip().split()
    if not tokens:
        return None
    command = tokens[0].upper()
    args = tokens[1:]
    match command:
        case "LS":
            return ls()
        case "MKDIR":
            return mkdir(args[0])
        case "TOUCH":
            return touch(args[0])
        case "FILTER":
            return filter_items(args[0])
    \end{minted}
\end{frame}

\begin{frame}[fragile]
    \frametitle{Practice Time: Multiple Commands Solution (Cont.)}

    Solution (Cont.):
    \begin{minted}{python}
def main():
    input_line = input().strip()
    commands = input_line.split("&&")
    results = list(map(processCommand, commands))
    results = [res for res in results if res is not None]
    print(results)
    \end{minted}
\end{frame}

\subsection{Exercise: Checking Commands}
\begin{frame}[fragile]
    \frametitle{Practice Time: Checking Commands}
    Modify your command sequence processor to
    handle input validation using filter and lambda functions.
    Remove any invalid commands before processing the sequence.

    Valid Command Patterns:
    \begin{itemize}
        \small
        \item \texttt{LS} 0 arguments
        \item \texttt{MKDIR <dirname>} 1 argument (must start with \python{"/"} and end with no suffix)
        \item \texttt{TOUCH <filename>} 1 argument (must not start with \python{"/"})
        \item \texttt{FILTER <pattern>} 1 argument (must start with \python{"."} or \python{"/"})
    \end{itemize}

    Requirements:
    \begin{itemize}
        \small
        \item Use \python{filter} and \python{lambda} to remove invalid commands
        \item Skip empty commands and commands with incorrect arguments
        \item Return outputs only for valid commands that were processed
        \item Maintain the original order of valid commands
        \item All list outputs should be sorted alphanumerically
    \end{itemize}
\end{frame}
\begin{frame}[fragile]
    \frametitle{Practice Time: Checking Commands (Cont.)}
    Example 1:
    \begin{itemize}
        \small
        \item Input: \texttt{MKDIR /src \&\& TOUCH \&\& LS \&\& FILTER}
        \item Output:
              \begin{minted}{python}
[
    "Directory '/src' created",
    ["/Desktop", "/Documents", "/home", "/src", "test.txt"]
]
              \end{minted}
    \end{itemize}
    Example 2:
    \begin{itemize}
        \item Input: \texttt{TOUCH app.py \&\& MKDIR docs \&\& LS}
        \item Output:
              \begin{minted}{python}
[
    "File 'app.py' created",
    [
        "/Desktop", "/Documents", "/home", 
        "app.py", "test.txt"
    ]
]
              \end{minted}
    \end{itemize}
\end{frame}
\begin{frame}[fragile]
    \frametitle{Practice Time: Checking Commands (Cont.)}
    Example 3:
    \begin{itemize}
        \small
        \item Input:
              \begin{minted}{text}
TOUCH invalid.py && MKDIR /invalid.py && 
TOUCH /invalid && FILTER .py && FILTER /
        \end{minted}
        \item Output:
              \begin{minted}{python}
[
    "File 'invalid.py' created",
    ["invalid.py"],
    ["/Desktop", "/Documents", "/home"]
]
              \end{minted}
    \end{itemize}
\end{frame}
\begin{frame}[fragile]
    \frametitle{Practice Time: Checking Commands Solution}

    Solution:
    \begin{minted}{python}
def isValidCommand(line):
    tokens = line.strip().split()
    if not tokens:
        return False
    args = tokens[1:]
    match tokens[0].upper():
        case "LS": return len(args) == 0
        case "MKDIR": return len(args) == 1 \
            and args[0].startswith("/") \
            and "." not in args[0]
        case "TOUCH": return len(args) == 1 \
            and not args[0].startswith("/")
        case "FILTER": return len(args) == 1 \
            and (args[0].startswith(".") or args[0] == "/")
    return False
    \end{minted}
\end{frame}

\begin{frame}[fragile]
    \frametitle{Practice Time: Checking Commands Solution (Cont.)}

    Solution (Cont.):
    \begin{minted}{python}
def main():
    input_line = input().strip()
    commands = input_line.split("&&")
    valid_commands = list(filter(isValidCommand, commands))
    results = list(map(processCommand, valid_commands))
    results = [res for res in results if res is not None]
    print(results)
    \end{minted}
\end{frame}

\section{Q\&A and Reminders}
\begin{frame}
    \frametitle{Any Questions?}

    Feel free to ask us about the tutorials and python coding!

    \begin{itemize}
        \item Contact us via email
        \item Ask us in person right now
        \item Post on Moodle or Ed
    \end{itemize}
\end{frame}

\begin{frame}
    \frametitle{Thank You!}

    Enjoy your coding journey in Python!

    \begin{itemize}
        \item Happy Coding!
        \item See you in the next tutorial (HW311, 10:00 - 11:50, Thursday)
        \item Submit your tutorial 7 exercises before \textbf{November 22 23:59}!
        \item Don't forget tutorial 6 exercises\\(due on \textbf{November 14 23:59})!
        \item Bonus: \LaTeX source code and compiled pdf of this slides available at
              \href{https://github.com/xtz206/COMP1117-Autumn25-Tutorials/releases}
              {https://github.com/xtz206/COMP1117-Autumn25-Tutorials/releases}
    \end{itemize}
\end{frame}

\end{document}