\documentclass{beamer}
\usetheme{Madrid}
\usecolortheme{default}

\usepackage{minted}

\newcommand{\python}[1]{\mintinline{python}|#1|}

\title{COMP1117 Tutorial 8}
\author{YUAN Wenxuan}
\date{November 13, 2025}

\begin{document}

\frame{\titlepage}

\begin{frame}
    \frametitle{Overview of Today's Tutorial}
    \tableofcontents
\end{frame}

\section{Recap and Extensions}
\subsection{Recap: Python String}
\begin{frame}
    \frametitle{Recap: String Index and Slicing}
    \begin{itemize}
        \item String is an immutable sequence of characters
        \item Note: in Python, there is no separate \texttt{char} type
        \item Syntactic sugar for negative index (just add length)
        \item Note: \python{IndexError} if \texttt{i < -len}
        \item Slicing syntax: \python{string[start:end:step]}
        \item Note: Omitting \texttt{start}, \texttt{end}, or \texttt{step} \\
              uses default values (0, length of string, 1 respectively)
    \end{itemize}
\end{frame}
\begin{frame}[fragile]
    \frametitle{Recap: String Methods}
    \begin{itemize}
        \item Boolean methods: \python{isalpha()}, \python{isdigit()}, \python{islower()}, \python{isupper()}, \python{isspace()}
        \item Searching methods: \python{find()}, \python{rfind()}, \python{index()}, \python{rindex()}
        \item Modification methods: \python{lower()}, \python{upper()}, \python{strip()}, \python{replace()}
        \item Splitting and joining: \python{split()}, \python{rsplit()}, \python{join()}
        \item Note: String methods do not modify the original string, \\
              they return a new string instead
        \item Visit Python String Methods for more details at
              \href{https://docs.python.org/3/library/stdtypes.html\#string-methods}
              {https://docs.python.org/3/library/stdtypes.html\#string-methods}
    \end{itemize}
\end{frame}
\subsection{Recap: File Operations}
\begin{frame}[fragile]
    \frametitle{Recap: File Operations}
    \begin{itemize}
        \small
        \item File operations in Python are performed using built-in functions
        \item Opening a file: \python{open(filename, mode)}
        \item Common modes: \texttt{'r'} (read), \texttt{'w'} (write), \texttt{'a'} (append), \texttt{'b'} (binary)
        \item Reading from a file: \python{file.read()}, \python{file.readline()}, \python{file.readlines()}
        \item Writing to a file: \python{file.write(str)}, \python{file.writelines(list[str])}
        \item Closing a file: \python{file.close()}
        \item Using \texttt{with} context manager and for automatic file closing:
        \begin{minted}{python}
with open('file.txt', 'r') as file:
    content = file.read()
        \end{minted}
        \item Using for-loop to read lines from a file (Memory efficient):
        \begin{minted}{python}
with open('file.txt', 'r') as file:
    for line in file:
        print(line)
        \end{minted}
    \end{itemize}
\end{frame}
\subsection{Recap: Error Handling}
\begin{frame}[fragile]
    \frametitle{Recap: Error Handling}
    \begin{itemize}
        \item In Tutorial 7, we have learned about error handling in Python
        \item Syntax(try-except-finally):
        \begin{minted}{python}
try:
    # code that may raise an exception
except SomeException as e:
    # code to handle the exception
finally:
    # code that will always execute
        \end{minted}
        \item You can have multiple \texttt{except} blocks to handle different exceptions
        \item The \texttt{finally} block is optional and will always execute, regardless of whether an exception occurred or not
        \item You can also use \texttt{raise} to raise exceptions manually
        \item Syntax(raise):
        \begin{minted}{python}
raise SomeException("Error message")
        \end{minted}
    \end{itemize}
\end{frame}

\subsection{Extension: Modules and Packages}
\begin{frame}[fragile]
    \frametitle{Extension: Modules and Packages (Optioanl)}
    \begin{itemize}
        \item Modules are files containing Python code (functions, classes, etc.).
        \item Create your own modules by saving code in a \texttt{.py} file
        \item Packages are directories containing multiple modules
        \item A package can contain a special file named \texttt{\_\_init\_\_.py} to initialize the package
        \item Using \python{import} to import modules or packages:
        \begin{minted}{python}
import package_name.module_name
from package_name import module_name as alias_name
from module_name import * # Should be avoided
        \end{minted}
        \item Advantages:
        \begin{itemize}
            \item Code organization
            \item Reusability
            \item Namespace management
        \end{itemize}
    \end{itemize}
\end{frame}

\section{Practice Time!}
\begin{frame}
    \frametitle{Time to Practice!}
    \begin{itemize}
        \item MCQs
        \item Factorial
        \item Combination
        \item Ackermann function
        \item Base conversion
    \end{itemize}
\end{frame}

\subsection{MCQs}
\begin{frame}[fragile]
    \frametitle{Practice Time: MCQ1}
    What is the output of the following code?
    \begin{minted}{python}
def f1(x):
    return x < -1
L1 = [1, -2, -3, 4, 5]
m1 = list(map(f1, L1))
print(m1)
    \end{minted}
    \begin{itemize}
        \item A. Error
        \item B. Address of \texttt{m1}
        \item C. \texttt{[-2, 3]}
        \item D. \texttt{[False, True, True, False, False]}
    \end{itemize}
\end{frame}
\begin{frame}[fragile]
    \frametitle{Practice Time: MCQ1 Solution}
    What is the output of the following code?
    \begin{minted}{python}
def f1(x):
    return x < -1
L1 = [1, -2, -3, 4, 5]
m1 = list(map(f1, L1))
print(m1)
    \end{minted}
    \begin{itemize}
        \item A. Error
        \item B. Address of \texttt{m1}
        \item C. \texttt{[-2, 3]}
        \item \alert{D. \texttt{[False, True, True, False, False]}}
    \end{itemize}

    Explanation:
    % TODO: fill in mcq 1 explanation
\end{frame}
\begin{frame}[fragile]
    \frametitle{Practice Time: MCQ2}
    What is the output of the following code?
    \begin{minted}{python}
def f2(x):
    return x < 2
L2 = [1, -2, -3, 4, 5]
m2 = list(filter(f2, L2))
print(m2)
    \end{minted}
    \begin{itemize}
        \item A. Error
        \item B. Address of \texttt{m2}
        \item C. \texttt{[1, -2, -3]}
        \item D. \texttt{[True, True, True, False, False]}
    \end{itemize}
\end{frame}
\begin{frame}[fragile]
    \frametitle{Practice Time: MCQ2 Solution}
    What is the output of the following code?
    \begin{minted}{python}
def f2(x):
    return x < 2
L2 = [1, -2, -3, 4, 5]
m2 = list(filter(f2, L2))
print(m2)
    \end{minted}
    \begin{itemize}
        \item A. Error
        \item B. Address of \texttt{m2}
        \item \alert{C. \texttt{[1, -2, -3]}}
        \item D. \texttt{[True, True, True, False, False]}
    \end{itemize}

    Explanation:
    % TODO: fill in mcq 2 explanation
\end{frame}
\begin{frame}[fragile]
    \frametitle{Practice Time: MCQ3}
    What is the output of the following code?
    \begin{minted}{python}
L3 = [-2, 4]
m3 = map(lambda x: x*2, L3)
print(m3)
    \end{minted}
    \begin{itemize}
        \item A. Error
        \item B. Map object representing \texttt{m3}
        \item C. \texttt{[-4, 8]}
        \item D. None of the above
    \end{itemize}
\end{frame}
\begin{frame}[fragile]
    \frametitle{Practice Time: MCQ3 Solution}
    What is the output of the following code?
    \begin{minted}{python}
L3 = [-2, 4]
m3 = map(lambda x: x*2, L3)
print(m3)
    \end{minted}
    \begin{itemize}
        \item A. Error
        \item \alert{B. Map object representing \texttt{m3}}
        \item C. \texttt{[-4, 8]}
        \item D. None of the above
    \end{itemize}

    Explanation:
    % TODO: fill in mcq 3 explanation
\end{frame}
\begin{frame}[fragile]
    \frametitle{Practice Time: MCQ4}
    What is the output of the following code?
    \begin{minted}{python}
print(list(map(
    (lambda x: x**2), 
    filter((lambda x: x%2==0), range(10))
)))
    \end{minted}
    \begin{itemize}
        \item A. \texttt{[0, 1, 2, 3, 4, 5, 6, 7, 8, 9]}
        \item B. \texttt{[0, 1, 4, 9, 16, 25, 36, 49, 64, 81]}
        \item C. \texttt{[0, 4, 16, 36, 64]}
        \item D. Error
    \end{itemize}
\end{frame}
\begin{frame}[fragile]
    \frametitle{Practice Time: MCQ4 Solution}
    What is the output of the following code?
    \begin{minted}{python}
print(list(map(
    (lambda x: x**2), 
    filter((lambda x: x%2==0), range(10))
)))
    \end{minted}
    \begin{itemize}
        \item A. \texttt{[0, 1, 2, 3, 4, 5, 6, 7, 8, 9]}
        \item B. \texttt{[0, 1, 4, 9, 16, 25, 36, 49, 64, 81]}
        \item \alert{C. \texttt{[0, 4, 16, 36, 64]}}
        \item D. Error
    \end{itemize}

    Explanation:
    % TODO: fill in mcq 4 explanation
\end{frame}
\begin{frame}[fragile]
    \frametitle{Practice Time: MCQ5}
    What is the output of the following code?
    \begin{minted}{python}
def fun(n):
    if (n > 100):
        return n - 5
    return fun(fun(n + 11))
print(fun(98))
    \end{minted}
    \begin{itemize}
        \item A. \texttt{98}
        \item B. \texttt{99}
        \item C. \texttt{100}
        \item D. \python{RecursionError}
    \end{itemize}
\end{frame}
\begin{frame}[fragile]
    \frametitle{Practice Time: MCQ5 Solution}
    What is the output of the following code?
    \begin{minted}{python}
def fun(n):
    if (n > 100):
        return n - 5
    return fun(fun(n + 11))
print(fun(98))
    \end{minted}
    \begin{itemize}
        \item A. \texttt{98}
        \item \alert{B. \texttt{99}}
        \item C. \texttt{100}
        \item D. \python{RecursionError}
    \end{itemize}

    Explanation:
    \begin{minted}{text}
fun(98) = fun(fun(109)) = fun(104) = 99
    \end{minted}
\end{frame}

\subsection{Exercise: Factorial}
\begin{frame}[fragile]
    \frametitle{Practice Time: Factorial}

    Write a program to calculate the factorial of
    a given number $n$ by using the recursive function.

    Here is a recursive formula for Factorial:
    $$
        n! = n \times (n - 1)! \text{ and } 0! = 1
    $$

    Remarks: For more information about Factorial, you may visit this website
    \href{https://en.wikipedia.org/wiki/Factorial}{https://en.wikipedia.org/wiki/Factorial}.

    Examples:
    \begin{center}
        \begin{tabular}{|c|c|}
            \hline
            Input & Output \\ \hline
            1     & 1      \\ \hline
            5     & 120    \\ \hline
        \end{tabular}
    \end{center}
\end{frame}
\begin{frame}[fragile]
    \frametitle{Practice Time: Factorial Solution}

    Solution:
    \begin{minted}{python}
def factorial(n):
    if n <= 0:
        return 1
    return n * factorial(n - 1)
    \end{minted}
\end{frame}

\subsection{Exercise: Combination}
\begin{frame}[fragile]
    \frametitle{Practice Time: Combination}

    In mathematics, combinations refer to the combination of
    $n$ things taken $k$ at a time without repetition.

    There is a recursive formula related to combinations:
    $$
        _nC_r = _{n-1}C_{r-1} + _{n-1}C_r \text{ and }
        _nC_0 = nC_n = 1, \ _nC_k = 0 \text{ for } k > n
    $$

    Write a Python program to calculate the value of
    $_nC_r$ of the given $n$ and $r$ by using the recursive function.

    Remarks: For more information about Combination, you may visit:
    \href{https://en.wikipedia.org/wiki/Combination}{https://en.wikipedia.org/wiki/Combination}.

    Examples (Newlines are replaced by spaces for simplicity):
    \begin{center}
        \begin{tabular}{|c|c|}
            \hline
            Input & Output \\ \hline
            6 3   & 20     \\ \hline
            8 1   & 8      \\ \hline
        \end{tabular}
    \end{center}
\end{frame}
\begin{frame}[fragile]
    \frametitle{Practice Time: Combination Solution}

    Solution:
    \begin{minted}{python}
def combination(n, r):
    if r == 0 or r == n:
        return 1
    if r > n:
        return 0
    return combination(n - 1, r - 1) + \
        combination(n - 1, r)
    \end{minted}
\end{frame}

\subsection{Exercise: Ackermann function}
\begin{frame}[fragile]
    \frametitle{Practice Time: Ackermann function}

    Implement Ackermann function by recursive function calls.
    Ackermann function is defined as follows:
    $$
        A(m, n) = \begin{cases}
            n + 1                 & \text{if } m = 0                    \\
            A(m - 1, 1)           & \text{if } m > 0 \text{ and } n = 0 \\
            A(m - 1, A(m, n - 1)) & \text{if } m > 0 \text{ and } n > 0
        \end{cases}
    $$

    Examples (Newlines are replaced by spaces for simplicity):
    \begin{center}
        \begin{tabular}{|c|c|}
            \hline
            Input & Output \\ \hline
            1 2   & 4      \\ \hline
            4 0   & 13     \\ \hline
        \end{tabular}
    \end{center}
\end{frame}
\begin{frame}[fragile]
    \frametitle{Practice Time: Ackermann function Solution}

    Solution:
    \begin{minted}{python}
def ackermann(m, n):
    if m == 0:
        return n + 1
    if n == 0:
        return ackermann(m - 1, 1)
    return ackermann(m - 1, ackermann(m, n - 1))
    \end{minted}
\end{frame}

\subsection{Exercise: Base conversion}
\begin{frame}[fragile]
    \frametitle{Practice Time: Base conversion}

    Complete the code below by defining a recursive function which
    converts a decimal integer $n$ to a string in base $b$.

    You may assume the input $n$ is an integer in decimal and
    the input $b$ is in the range 2 to 16, inclusively.

    \begin{minted}{python}
    n = int(input())
    b = int(input())
    print(convert_base(n, b))
    \end{minted}


    Examples (Newlines are replaced by spaces for simplicity):
    \begin{center}
        \begin{tabular}{|c|c|}
            \hline
            Input   & Output \\ \hline
            13 2    & 1101   \\ \hline
            56 4    & 320    \\ \hline
            1117 8  & 2135   \\ \hline
            2019 16 & 7E3    \\ \hline
        \end{tabular}
    \end{center}
\end{frame}
\begin{frame}[fragile]
    \frametitle{Practice Time: Base conversion Solution}

    Solution:
    \begin{minted}{python}
def convert_base(n: int, b: int) -> str:
    # Check special cases (You may skip this part)
    if b < 2 or b > 16:
        raise ValueError("Base must be between 2 and 16")
    if n < 0:
        return "-" + convert_base(-n, b)
    
    if n == 0:
        return "0"

    digits = "0123456789ABCDEF"
    if n < b:
        return digits[n]
    return convert_base(n // b, b) + digits[n % b]
    \end{minted}
\end{frame}

\section{Q\&A and Reminders}
\begin{frame}
    \frametitle{Any Questions?}

    Feel free to ask us about the tutorials and python coding!

    \begin{itemize}
        \item Contact us via email
        \item Ask us in person right now
        \item Post on Moodle or Ed
    \end{itemize}
\end{frame}

\begin{frame}
    \frametitle{Thank You!}

    Enjoy your coding journey in Python!

    \begin{itemize}
        \item Happy Coding!
        \item See you in the next tutorial (HW311, 10:00 - 11:50, Thursday)
        \item Submit your tutorial 7 and 8 exercises before \textbf{November 21 23:59}!
        \item Don't forget tutorial 6 exercises (due on \textbf{TOMORROW 23:59})!
        \item Bonus: \LaTeX source code and compiled pdf of this slides available at
              \href{https://github.com/xtz206/COMP1117-Autumn25-Tutorials/releases}
              {https://github.com/xtz206/COMP1117-Autumn25-Tutorials/releases}
    \end{itemize}
\end{frame}

\end{document}