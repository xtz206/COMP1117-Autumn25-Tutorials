\documentclass{beamer}
\usetheme{Madrid}
\usecolortheme{default}

\usepackage{minted}
\usepackage{makecell}

\newcommand{\python}[1]{\mintinline{python}|#1|}

\title{COMP1117 Tutorial 5}
\author{YUAN Wenxuan}
\date{October 23, 2025}

\begin{document}

\frame{\titlepage}

\begin{frame}
    \frametitle{Overview of Today's Tutorial}
    \tableofcontents
\end{frame}

\section{Recap and Extensions}
\subsection{Recap: TODO1RC}
\begin{frame}
    \frametitle{Recap: TODO1RC}
    \begin{itemize}
        % TODO: Recap items
        \item
    \end{itemize}
\end{frame}
\subsection{Recap: TODO2RC}
\begin{frame}
    \frametitle{Recap: TODO2RC}
    \begin{itemize}
        % TODO: Recap items
        \item
    \end{itemize}
\end{frame}
\subsection{Recap: TODO3RC}
\begin{frame}
    \frametitle{Recap: TODO3RC}
    \begin{itemize}
        % TODO: Recap items
        \item
    \end{itemize}
\end{frame}

\subsection{Extension: TODO1ET}
\begin{frame}
    \frametitle{Extension: TODO1ET}
    \begin{itemize}
        % TODO: Extension items
        \item
    \end{itemize}
\end{frame}
\subsection{Extension: TODO2ET}
\begin{frame}
    \frametitle{Extension: TODO2ET}
    \begin{itemize}
        % TODO: Extension items
        \item
    \end{itemize}
\end{frame}
\subsection{Extension: TODO3ET}
\begin{frame}
    \frametitle{Extension: TODO3ET}
    \begin{itemize}
        % TODO: Extension items
        \item
    \end{itemize}
\end{frame}
\subsection{Extension: TODO4ET}
\begin{frame}
    \frametitle{Extension: TODO4ET}
    \begin{itemize}
        % TODO: Extension items
        \item
    \end{itemize}
\end{frame}

\section{Practice Time!}
\begin{frame}
    \frametitle{Time to Practice!}
    \begin{itemize}
        \item MCQs
        \item Inverted Triangle
        \item Special Triangle
        \item Sum of List
        \item Set of Numbers
    \end{itemize}
\end{frame}

\subsection{MCQs}
\begin{frame}[fragile]
    \frametitle{Practice Time: MCQ1}
    What is the output of the following code?
    \begin{minted}{python}
    fruits = ["apple", "banana", "cherry"]
    for x in fruits:
    if x == "banana":
        break
    print(x, end=' ')    
    \end{minted}
    \begin{itemize}
        \item A. \texttt{apple banana cherry}
        \item B. \texttt{apple banana}
        \item C. \texttt{apple }
        \item D. None of the above
    \end{itemize}
\end{frame}
\begin{frame}[fragile]
    \frametitle{Practice Time: MCQ1 Solution}
    What is the output of the following code?

    \begin{minted}{python}
    fruits = ["apple", "banana", "cherry"]
    for x in fruits:
    if x == "banana":
        break
    print(x, end=' ')    
    \end{minted}

    \begin{itemize}
        \item A. \texttt{apple banana cherry}
        \item B. \texttt{apple banana}
        \item \alert{C. \texttt{apple }}
        \item D. None of the above
    \end{itemize}

    Explanation:

    When \python{"banana"} is met in the iteration,
    the \python{break} statement exits the loop immediately.
    Only \python{"apple"} is printed before the loop ends.
\end{frame}
\begin{frame}[fragile]
    \frametitle{Practice Time: MCQ2}
    What is the output of the following code?

    \begin{minted}{python}
    fruits = {'a': "apple", 'b': "banana", 'c': "cherry"}
    for x in fruits:
        if x == "banana":
            break
        print(x, end=' ')
    \end{minted}

    \begin{itemize}
        \item A. \texttt{a b c }
        \item B. \texttt{apple banana cherry }
        \item C. \texttt{a }
        \item D. \texttt{apple }
    \end{itemize}
\end{frame}
\begin{frame}[fragile]
    \frametitle{Practice Time: MCQ2 Solution}
    What is the output of the following code?

    \begin{minted}{python}
    fruits = {'a': "apple", 'b': "banana", 'c': "cherry"}
    for x in fruits:
        if x == "banana":
            break
        print(x, end=' ')
    \end{minted}

    \begin{itemize}
        \item \alert{A. \texttt{a b c }}
        \item B. \texttt{apple banana cherry }
        \item C. \texttt{a }
        \item D. \texttt{apple }
    \end{itemize}

    Explanation:
    The loop iterates over the \textbf{keys} of the dictionary.
    Since none of the keys is \python{"banana"},
    the loop prints all the keys: \python{"a b c "}.
\end{frame}
\begin{frame}[fragile]
    \frametitle{Practice Time: MCQ3}
    What is the output of the following code?

    \begin{minted}{python}
    for i in range(1, 7):
        for j in range(1, 7):
            if i == j:
                print(i+j, end=' ')
            else:
                print('-', end=' ')
        print()
    \end{minted}

    \begin{tabular}{cl|cl}
        A & \texttt{2 4 6 8 10 12} & B & \texttt{2 4 6 8 10 12 14} \\\hline
        C &
        \makecell[tl]{
        \texttt{2 - - - - - }                                      \\
        \texttt{- 4 - - - - }                                      \\
        \texttt{- - 6 - - - }                                      \\
        \texttt{- - - 8 - - }                                      \\
        \texttt{- - - - 10 - }                                     \\
        \texttt{- - - - - 12}                                      \\
        }
          & D                      &
        \makecell[tl]{
        \texttt{2 - - - - - - }                                    \\
        \texttt{- 4 - - - - - }                                    \\
        \texttt{- - 6 - - - - }                                    \\
        \texttt{- - - 8 - - - }                                    \\
        \texttt{- - - - 10 - - }                                   \\
        \texttt{- - - - - 12 - }                                   \\
        \texttt{- - - - - - 14}                                    \\
        }
    \end{tabular}
\end{frame}
\begin{frame}[fragile]
    \frametitle{Practice Time: MCQ3 Solution}
    What is the output of the following code?

    \begin{minted}{python}
    for i in range(1, 7):
        for j in range(1, 7):
            if i == j:
                print(i+j, end=' ')
            else:
                print('-', end=' ')
        print()
    \end{minted}
    \begin{itemize}
        \item \alert{C.}
              \alert{
                  \makecell[tl]{
                      \texttt{2 - - - - - }                                              \\
                      \texttt{- 4 - - - - }                                              \\
                      \texttt{- - 6 - - - }                                              \\
                      \texttt{- - - 8 - - }                                              \\
                      \texttt{- - - - 10 - }                                             \\
                      \texttt{- - - - - 12}                                              \\
                  }}
    \end{itemize}
    Explanation: Both loops iterates from 1 to 6.
\end{frame}
\begin{frame}[fragile]
    \frametitle{Practice Time: MCQ4}
    What is the output of the following code?

    \begin{minted}{python}
    for i in range(5, 0, -1):
        for j in range(i, 0, -1):
            print(2*j-1, end=' ')
        print()
    \end{minted}

    \begin{tabular}{|c|c|c|c|}
        \hline
        A & B & C & D          \\\hline
        \makecell[tl]{
        \texttt{9 7 5 3 1 }    \\
        \texttt{9 7 5 3 }      \\
        \texttt{9 7 5 }        \\
        \texttt{9 7 }          \\
        \texttt{9 }            \\
        }
          &
        \makecell[tl]{
        \texttt{9 7 5 3 1 }    \\
        \texttt{7 5 3 1}       \\
        \texttt{5 3 1 }        \\
        \texttt{3 1 }          \\
        \texttt{1 }            \\
        }
          &
        \makecell[tl]{
        \texttt{9 7 5 3 1 }    \\
        \texttt{7 5 3 1 -1}    \\
        \texttt{5 3 1 -1 -3}   \\
        \texttt{3 1 -1 -3 -5}  \\
        \texttt{1 -1 -3 -5 -7} \\
        }
          &
        \makecell[tl]{
        \texttt{1 }            \\
        \texttt{1 3 }          \\
        \texttt{1 3 5 }        \\
        \texttt{1 3 5 7 }      \\
        \texttt{1 3 5 7 9 }    \\
        }                      \\\hline
    \end{tabular}
\end{frame}
\begin{frame}[fragile]
    \frametitle{Practice Time: MCQ4 Solution}
    What is the output of the following code?

    \begin{minted}{python}
    for i in range(5, 0, -1):
        for j in range(i, 0, -1):
            print(2*j-1, end=' ')
        print()
    \end{minted}

    \begin{tabular}{|c|c|c|c|}
        \hline
        A & \alert{B} & C & D  \\\hline
        \makecell[tl]{
        \texttt{9 7 5 3 1 }    \\
        \texttt{9 7 5 3 }      \\
        \texttt{9 7 5 }        \\
        \texttt{9 7 }          \\
        \texttt{9 }            \\
        }
          &
        \alert{\makecell[tl]{
        \texttt{9 7 5 3 1 }    \\
        \texttt{7 5 3 1}       \\
        \texttt{5 3 1 }        \\
        \texttt{3 1 }          \\
        \texttt{1 }            \\
            }}
          &
        \makecell[tl]{
        \texttt{9 7 5 3 1 }    \\
        \texttt{7 5 3 1 -1}    \\
        \texttt{5 3 1 -1 -3}   \\
        \texttt{3 1 -1 -3 -5}  \\
        \texttt{1 -1 -3 -5 -7} \\
        }
          &
        \makecell[tl]{
        \texttt{1 }            \\
        \texttt{1 3 }          \\
        \texttt{1 3 5 }        \\
        \texttt{1 3 5 7 }      \\
        \texttt{1 3 5 7 9 }    \\
        }                      \\\hline
    \end{tabular}

    Explanation:

    The outer loop iterates with \python{i} from 5 down to 1.
    The inner loop iterates with \python{j} from \python{i} down to 1.
    Every line has decreasing odd numbers starting from 9.

\end{frame}
\begin{frame}[fragile]
    \frametitle{Practice Time: MCQ5}
    What is the output of the following code?

    \begin{small}
        \begin{minted}{python}
    p = 1
    while p < 6:
        q = 1
        while q < 6:
            print(p, q, sep=', ')
            q += 1
            if q == 2:
                break
        print("****")
        p += 1
        if p == 3:
            break
        \end{minted}

        \begin{tabular}{|c|c|c|c|}
            \hline
            A & B & C & D \\\hline
            \texttt{1, 1}
              &
            \makecell[tl]{
            \texttt{1, 1} \\
            \texttt{****} \\
            }
              &
            \makecell[tl]{
            \texttt{1, 1} \\
            \texttt{****} \\
            \texttt{2, 1} \\
            \texttt{****} \\
            }
              &
            None of the above
            \\\hline
        \end{tabular}
    \end{small}
\end{frame}
\begin{frame}[fragile]
    \frametitle{Practice Time: MCQ5 Solution}
    What is the output of the following code?
    \begin{small}
        \begin{minted}{python}
    p = 1
    while p < 6:
        q = 1
        while q < 6:
            print(p, q, sep=', ')
            q += 1
            if q == 2:
                break
        print("****")
        p += 1
        if p == 3:
            break
        \end{minted}

        \begin{tabular}{|c|c|c|c|}
            \hline
            A & B & \alert{C} & D \\\hline
            \texttt{1, 1}
              &
            \makecell[tl]{
            \texttt{1, 1}         \\
            \texttt{****}         \\
            }
              &
            \alert{\makecell[tl]{
            \texttt{1, 1}         \\
            \texttt{****}         \\
            \texttt{2, 1}         \\
            \texttt{****}         \\
                }}
              &
            None of the above
            \\\hline
        \end{tabular}
    \end{small}
\end{frame}

\subsection{Exercise: Inverted Triangle}
\begin{frame}[fragile]
    \frametitle{Practice Time: Inverted Triangle}

    Write a program to ask the user to input a positive integer $n$.
    Use \python{for} loop to finish this question.
    Print an inverted triangle as follows.
    \begin{minted}{text}
    0 1 2 ... n-1 n
    0 1 2 ... n-1
    ...
    0 1 2
    0 1
    0
    \end{minted}

    Examples:
    \begin{center}
        \begin{tabular}{|c|c|}
            \hline
            Input & Output                     \\ \hline
            1     & \makecell[tl]{\texttt{0 1} \\ \texttt{0}}       \\ \hline
            3     &
            \makecell[tl]{\texttt{0 1 2 3}     \\ \texttt{0 1 2} \\ \texttt{0 1} \\ \texttt{0}}
            \\ \hline
        \end{tabular}
    \end{center}
\end{frame}
\begin{frame}[fragile]
    \frametitle{Practice Time: Inverted Triangle Solution}

    Solution:
    \begin{minted}{python}
    n = int(input())
    for i in range(n + 1):
        for j in range(n - i):
            if j < n - i - 1:
                print(j, end=' ')
            else:
                print(j)
    \end{minted}
\end{frame}

\subsection{Exercise: Special Triangle}
\begin{frame}[fragile]
    \frametitle{Practice Time: Special Triangle}

    Write a program to ask the user to input a positive integer $n$.
    Use \python{for} loop to finish this question.
    Print a triangle with $n$ lines as follows.
    \begin{minted}{text}
    0
    1 0
    0 1 0
    ...
    \end{minted}

    Examples:

    \begin{center}
        \begin{tabular}{|c|c|}
            \hline
            Input & Output     \\ \hline
            1     & 0          \\ \hline
            5     &
            \makecell[tl]{
            \texttt{0}         \\
            \texttt{1 0}       \\
            \texttt{0 1 0}     \\
            \texttt{1 0 1 0}   \\
            \texttt{0 1 0 1 0} \\
            }
            \\ \hline
        \end{tabular}
    \end{center}
\end{frame}
\begin{frame}[fragile]
    \frametitle{Practice Time: Special Triangle Solution}

    Solution:

    \begin{minted}{python}
    n = int(input())
    for i in range(n):
        for j in range(i + 1):
            if j < i:
                print((i + j) % 2, end=' ')
            else:
                print((i + j) % 2)
    \end{minted}
\end{frame}

\subsection{Exercise: Sum of List}
\begin{frame}[fragile]
    \frametitle{Practice Time: Sum of List}
    Complete the code below by defining the two functions
    \python{readList()} and \python{calSum()} as explained below.

    \begin{minted}{python}
    myList = readList()
    sum = calSum(myList)
    print(sum)
    \end{minted}

    \begin{itemize}
        \item \python{readList()}:
              Read numbers from user until zero is received,
              return the list of input numbers.
        \item \python{calSum(myList)}:
              Calculate the sum of the values in the list and return it.
        \item Hints: Use \python{list.append(value)} to add an element to the list.
    \end{itemize}

\end{frame}
\begin{frame}[fragile]
    \frametitle{Practice Time: Sum of List Solution}

    Solution:
    \begin{minted}{python}
    def readList():
        ret = []
        while True:
            n = int(input())
            if n == 0:
                return ret
            ret.append(n)
    
    def calSum(myList):
        ret = 0
        for num in myList:
            ret += num
        return ret
    \end{minted}
\end{frame}

\subsection{Exercise: Set of Numbers}
\begin{frame}[fragile]
    \frametitle{Practice Time: Set of Numbers}

    Complete the code below by defining the two functions
    \python{readSet()} and \python{findNo()} as explained below.

    \begin{minted}{python}
    mySet = readSet()
    number = int(input("Find number?"))
    print(findNo(mySet, number))
    \end{minted}

    \begin{itemize}
        \item \python{readSet()}:
              Read numbers from user until zero is received,
              return the set of input numbers.
        \item \python{findNo(mySet, number)}:
              return \python{True} if \python{number} is in \python{mySet},
              otherwise return \python{False}.
        \item Hints: You can ignore the case of entering multiple identical numbers.
    \end{itemize}
\end{frame}
\begin{frame}[fragile]
    \frametitle{Practice Time: Set of Numbers Solution}

    Solution:
    \begin{minted}{python}
    def readSet():
        ret = set()
        while True:
            n = int(input())
            if n == 0:
                return ret
            ret.add(n)
    
    def findNo(mySet, number):
        return number in mySet
    \end{minted}
\end{frame}

\section{Q\&A and Reminders}
\begin{frame}
    \frametitle{Any Questions?}

    Feel free to ask us about the tutorials and python coding!

    \begin{itemize}
        \item Contact us via email
        \item Ask us in person right now
        \item Post on Moodle or Ed
    \end{itemize}
\end{frame}

\begin{frame}
    \frametitle{Thank You!}

    Enjoy your coding journey in Python!

    \begin{itemize}
        \item Happy Coding!
        \item See you in the next tutorial (HW311, 10:00 - 11:50, Thursday)
        \item Submit your tutorial 5 exercises before \textbf{October 31 23:59}!

        \item Bonus: \LaTeX source code and compiled pdf of this slides available at
              \href{https://github.com/xtz206/COMP1117-Autumn25-Tutorials/releases}
              {https://github.com/xtz206/COMP1117-Autumn25-Tutorials/releases}
    \end{itemize}
\end{frame}

\end{document}