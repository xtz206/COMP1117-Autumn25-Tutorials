\documentclass{beamer}
\usetheme{Madrid}
\usecolortheme{default}

\usepackage{minted}

\newcommand{\python}[1]{\mintinline{python}|#1|}

\title{COMP1117 Tutorial 10}
\author{YUAN Wenxuan}
\date{November 27, 2025}

\begin{document}

\frame{\titlepage}

\begin{frame}
    \frametitle{Overview of Today's Tutorial}
    \tableofcontents
\end{frame}

\section{Recap and Extensions}
\subsection{Recap: TODO1RC}
\begin{frame}
    \frametitle{Recap: TODO1RC}
    \begin{itemize}
        % TODO: Recap items
        \item
    \end{itemize}
\end{frame}
\subsection{Recap: TODO2RC}
\begin{frame}
    \frametitle{Recap: TODO2RC}
    \begin{itemize}
        % TODO: Recap items
        \item
    \end{itemize}
\end{frame}
\subsection{Recap: TODO3RC}
\begin{frame}
    \frametitle{Recap: TODO3RC}
    \begin{itemize}
        % TODO: Recap items
        \item
    \end{itemize}
\end{frame}

\subsection{Extension: TODO1ET}
\begin{frame}
    \frametitle{Extension: TODO1ET}
    \begin{itemize}
        % TODO: Extension items
        \item
    \end{itemize}
\end{frame}
\subsection{Extension: TODO2ET}
\begin{frame}
    \frametitle{Extension: TODO2ET}
    \begin{itemize}
        % TODO: Extension items
        \item
    \end{itemize}
\end{frame}

\section{Practice Time!}
\begin{frame}
    \frametitle{Time to Practice!}
    \begin{itemize}
        \item Personal Diary
        \item Simple User Registration System
        \item File Content Copier
        \item Robust Grade Converter
    \end{itemize}
\end{frame}


\subsection{Exercise: Personal Diary}
\begin{frame}[fragile]
    \frametitle{Practice Time: Personal Diary}

    Background:
    Create a simple diary program that allows a user to record their thoughts
    for the day. Each new entry should be saved without overwriting previous entries.

    Task: Write a Python program that:
    \begin{enumerate}
        \item Prompts the user to enter the current date (e.g., "2023-12-05").
        \item Prompts the user to enter a single-line diary entry.
        \item Formats the user's input into the string: \python{"[Date]: [Diary Entry]"}.
        \item Appends the formatted string to a file named \texttt{diary.txt}.
        \item Each time the program is run, the new diary entry should
              be added to the end of the file.
        \item The program should automatically create \texttt{diary.txt} if it does not already exist.
    \end{enumerate}
\end{frame}
\begin{frame}[fragile]
    \frametitle{Practice Time: Personal Diary (Cont.)}
    Example Interaction:
    \begin{minted}{text}
Enter today's date (YYYY-MM-DD):
2023-12-05
What's on your mind today?
Learned about file I/O in Python. It was fun!
    \end{minted}
    After the first run, \texttt{diary.txt} should contain:
    \begin{minted}{text}
[2023-12-05]: Learned about file I/O in Python. It was fun!
    \end{minted}
    After a second run, it might look like:
    \begin{minted}{text}
[2023-12-05]: Learned about file I/O in Python. It was fun!
[2023-12-06]: Practiced exception handling. 
    The try-except block is very useful.
    \end{minted}
\end{frame}
\begin{frame}[fragile]
    \frametitle{Practice Time: Personal Diary Solution}

    Solution:
    \begin{minted}{python}
def main():
    date = input("Enter today's date (YYYY-MM-DD): ")
    entry = input("What's on your mind today? ")
    result = f"[{date}]: {entry}\n"
    with open("diary.txt", "a") as file:
        file.write(result)
    print("Diary entry saved successfully.")
main()
    \end{minted}
\end{frame}

\subsection{Exercise: Simple User Registration System}
\begin{frame}[fragile]
    \frametitle{Practice Time: Simple User Registration System}
    Background:
    Simulate a simple user registration system
    where new usernames are stored in a file.
    To prevent duplicate accounts, the program must check if a username
    already exists before adding a new one.

    Tasks: Write a Python program to manage a user list in file
    name \texttt{users.txt} (one username per line).
    \begin{enumerate}
        \small
        \item Prompt a new user to enter their desired username.
        \item Read the entire \texttt{users.txt} file to check if
              the new username already exists.
              Be careful to handle the newline characters
              \python{"\n"} properly.
              at the end of each line for accurate matching.
        \item If the username \textbf{already exists}, print the message:
              \python{"Username '[username]' already exists."}
              \python{"Please choose another one."}.
        \item If the username \textbf{does not exist}, append the new username
              to the end of \texttt{users.txt} and print a success message:
              \python{"Username '[username]' registered successfully."}
        \item \textbf{Exception Handling:} The first time the program runs,
              \texttt{users.txt} may not exist.
              The program should handle this case without error and
              proceed to register the new user (as any username will be unique).
    \end{enumerate}
\end{frame}
\begin{frame}[fragile]
    \frametitle{Practice Time: Simple User Registration System (Cont.)}
    Example Interaction:

    First Run (when \texttt{users.txt} does not exist):
    \begin{minted}{text}
Enter a username to register: admin
User 'admin' registered successfully.
    \end{minted}
    Second Run (registering an existing username):
    \begin{minted}{text}
Enter a username to register: admin
Username 'admin' already exists. Please choose another one.
    \end{minted}
    Third Run (registering a new username):
    \begin{minted}{text}
Enter a username to register: guest
User 'guest' registered successfully.
    \end{minted}
\end{frame}
\begin{frame}[fragile]
    \frametitle{Practice Time: Simple User Registration System Solution}

    Solution:
    \begin{minted}[fontsize=\small]{python}
def main():
    filename = "users.txt"
    username = input("Enter a username to register: ")
    try:
        with open(filename, 'r') as file:
            if username in (line.strip() for line in file):
                print(f"Username '{username}' already exists. "
                      f"Please choose another one.")
                return
    except FileNotFoundError:
        pass
    with open(filename, 'a') as file:
        file.write(username + '\n')
    print(f"User '{username}' registered successfully.")
    main()
    \end{minted}
\end{frame}

\subsection{Exercise: File Content Copier}
\begin{frame}[fragile]
    \frametitle{Practice Time: File Content Copier}

    Background:
    Copying a file is a fundamental computer operation.
    We need to write a program that accurately duplicates
    the entire content of one file into a new file.

    Tasks: Write a Python program that:
    \begin{enumerate}
        \small
        \item First, manually create a file named \texttt{source.txt}
              and write a few lines of text into it.
        \item When the program runs, it should read
              the entire content of \texttt{source.txt}.
        \item The program will then write this content
              into a new file named \texttt{destination.txt}.
        \item If \texttt{destination.txt} already exists,
              its content should be completely overwritten.
        \item \textbf{Exception Handling:}
              If \texttt{source.txt} does not exist (\python{FileNotFoundError}),
              the program should not crash. Instead, it should print the error message:
              \python{"Error: source.txt not found. Connot perform copy."} and exit gracefully.
        \item Use the \python{with} statement to ensure files
              are automatically closed after the operations.
    \end{enumerate}

\end{frame}
\begin{frame}[fragile]
    \frametitle{Practice Time: File Content Copier Solution}

    Solution:
    \begin{minted}[fontsize=\small]{python}
def main():
    source_filename = "source.txt"
    destination_filename = "destination.txt"
    try:
        with open(source_filename, 'r') as source_file:
            content = source_file.read()
        with open(destination_filename, 'w') as dest_file:
            dest_file.write(content)
        print(f"Successfully copied content from "
              f"'{source_filename}' to '{destination_filename}'.")
    except FileNotFoundError:
        print("Error: source.txt not found. "
              "Cannot perform copy.")
main()
\end{minted}
\end{frame}

\subsection{Exercise: Robust Grade Converter}
\begin{frame}[fragile]
    \frametitle{Practice Time: Robust Grade Converter}
    Background:
    We need a tool to convert a numerical score (0-100) into a
    letter grade (A, B, C, D, F).
    This tool must be robust enough
    to handle various invalid user inputs,
    such as text or out-of range numbers.

    Tasks: Write a Python program that:
    \begin{enumerate}
        \small
        \item Continuously prompts the user to enter a score between 0 and 100 in a loop.
        \item \textbf{Exception Handling:} Use a \python{try-except} block to
              handle the user's input. If the input is not a valid number,
              the program should catch the \python{ValueError}, and print
              \python{"Invalid input. Please enter a number."} and continue the loop.
        \item If the input is a valid number, check if it is within the range of 0 to 100.
              If the score is \textbf{out of range} (less than 0 or greater than 100),
              print \python{"Error: Score must be between 0 and 100."}.
        \item If the score is valid and in range, print the corresponding letter grade
              according to the rules below, and then \textbf{break the loop} to end the program:
              90-100: A, 80-89: B, 70-79: C, 60-69: D, 0-59: F.
    \end{enumerate}
\end{frame}
\begin{frame}[fragile]
    \frametitle{Practice Time: Robust Grade Converter Solution}

    Solution:
    \begin{minted}[fontsize=\tiny]{python}
def main():
    while True:
        text = input("Enter a score (0-100): ")
        try:
            score = float(text)
            if score < 0 or score > 100:
                print("Error: Score must be between 0 and 100.")
                continue
            if score >= 90:
                grade = 'A'
            elif score >= 80:
                grade = 'B'
            elif score >= 70:
                grade = 'C'
            elif score >= 60:
                grade = 'D'
            else:
                grade = 'F'
            print(f"Grade: {grade}")
            break
        except ValueError:
            print("Invalid input. Please enter a number.")
main()
    \end{minted}
\end{frame}

\section{Q\&A and Reminders}
\begin{frame}
    \frametitle{Any Questions?}

    Feel free to ask us about the tutorials and python coding!

    \begin{itemize}
        \item Contact us via email
        \item Ask us in person right now
        \item Post on Moodle or Ed
    \end{itemize}
\end{frame}

\begin{frame}
    \frametitle{Thank You!}

    Enjoy your coding journey in Python!

    \begin{itemize}
        \item Happy Coding!
        \item See you in the next tutorial (HW311, 10:00 - 11:50, Thursday)
        \item Submit your tutorial 10 exercises before \textbf{December 1 23:59}!
        \item Don't forget tutorial 9 exercises\\(due on \textbf{TOMORROW November 28 23:59})!
        \item Bonus: \LaTeX source code and compiled pdf of this slides available at
              \href{https://github.com/xtz206/COMP1117-Autumn25-Tutorials/releases}
              {https://github.com/xtz206/COMP1117-Autumn25-Tutorials/releases}
    \end{itemize}
\end{frame}

\end{document}