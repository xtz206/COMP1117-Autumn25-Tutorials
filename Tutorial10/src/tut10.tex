\documentclass{beamer}
\usetheme{Madrid}
\usecolortheme{default}

\usepackage{minted}

\newcommand{\python}[1]{\mintinline{python}|#1|}

\title{COMP1117 Tutorial 10}
\author{YUAN Wenxuan}
\date{November 27, 2025}

\begin{document}

\frame{\titlepage}

\begin{frame}
    \frametitle{Overview of Today's Tutorial}
    \tableofcontents
\end{frame}

\section{Recap and Extensions}
\subsection{Recap: More on Containers}
\begin{frame}
    \frametitle{Recap: Dictionary}
    \begin{itemize}
        \item Use \python{dict.get(key, default)} to safely access values
              without raising \python{KeyError}.
        \item Use \python{dict.keys()}, \python{dict.values()},
              and \python{dict.items()} to iterate over keys, values,
              and key-value pairs respectively.
        \item Use \python{in} keyword to check for key existence in a dictionary.
        \item Use \python{del} keyword to remove an item from a dictionary by key.
    \end{itemize}
\end{frame}
\begin{frame}
    \frametitle{Recap: Tuple}
    \begin{itemize}
        \item Tuples are immutable sequences, defined using parentheses \python{()}.
        \item Use indexing and slicing to access elements in a tuple.
        \item Tuples can be used as keys in dictionaries due to their immutability.
        \item Packing and unpacking can be used to assign multiple values at once.
        \item Try to modify a tuple will raise \python{TypeError}.
    \end{itemize}
\end{frame}
\subsection{Recap: Packing and Unpacking}
\begin{frame}
    \frametitle{Recap: Packing and Unpacking}
    \begin{itemize}
        \item Packing: grouping multiple values into a single tuple, e.g. \python{t = 1, 2, 3}.
        \item Unpacking: assign elements of a sequence to variables, e.g. \python{a, b = (1, 2)}.
        \item Extended unpacking: grab remaining items with \python{*}, e.g. \python{head, *tail = [1,2,3,4]}.
        \item Ignore values with underscore, e.g. \python{a, _, c = (1, 2, 3)}.
        \item Swap variables succinctly: \python{a, b = b, a}.
        \item Use unpacking in loops and function calls, e.g. \python{for x, y in pairs: ...} and \python{f(*args, **kwargs)}.
        \item Packing/unpacking works with lists, tuples and other iterables; types must match the number of targets (unless using \python{*}).
    \end{itemize}
\end{frame}

\subsection{Extension: OS Module}
\begin{frame}
    \frametitle{Extension: OS Module}
    \begin{itemize}
        \item Use \python{import os} to access operating system functionalities.
        \item Common functions:
              \begin{itemize}
                  \item \python{os.getcwd()}: Get current working directory.
                  \item \python{os.chdir(path)}: Change current working directory.
                  \item \python{os.listdir(path)}: List files and directories in the specified path.
                  \item \python{os.mkdir(path)}: Create a new directory.
                  \item \python{os.remove(path)}: Delete a file.
                  \item \python{os.rmdir(path)}: Remove an empty directory.
                  \item \python{os.path.join(path, *paths)}: Join multiple path components.
                  \item \python{os.path.exists(path)}: Check if a path exists.
              \end{itemize}
        \item Useful for file and directory management tasks.
    \end{itemize}
\end{frame}

\section{Practice Time!}
\begin{frame}
    \frametitle{Time to Practice!}
    \begin{itemize}
        \item Personal Diary
        \item Simple User Registration System
        \item Robust Grade Converter
        \item Palindrome
        \item Widest Fragment in a Sequence
    \end{itemize}
\end{frame}


\subsection{Exercise: Personal Diary}
\begin{frame}[fragile]
    \frametitle{Practice Time: Personal Diary}

    Background:
    Create a simple diary program that allows a user to record their thoughts
    for the day. Each new entry should be saved without overwriting previous entries.

    Task: Write a Python program that:
    \begin{enumerate}
        \item Prompts the user to enter the current date (e.g., "2023-12-05").
        \item Prompts the user to enter a single-line diary entry.
        \item Formats the user's input into the string: \python{"[Date]: [Diary Entry]"}.
        \item Appends the formatted string to a file named \texttt{diary.txt}.
        \item Each time the program is run, the new diary entry should
              be added to the end of the file.
        \item The program should automatically create \texttt{diary.txt} if it does not already exist.
    \end{enumerate}
\end{frame}
\begin{frame}[fragile]
    \frametitle{Practice Time: Personal Diary (Cont.)}
    Example Interaction:
    \begin{minted}{text}
Enter today's date (YYYY-MM-DD):
2023-12-05
What's on your mind today?
Learned about file I/O in Python. It was fun!
    \end{minted}
    After the first run, \texttt{diary.txt} should contain:
    \begin{minted}{text}
[2023-12-05]: Learned about file I/O in Python. It was fun!
    \end{minted}
    After a second run, it might look like:
    \begin{minted}{text}
[2023-12-05]: Learned about file I/O in Python. It was fun!
[2023-12-06]: Practiced exception handling. 
    The try-except block is very useful.
    \end{minted}
\end{frame}
\begin{frame}[fragile]
    \frametitle{Practice Time: Personal Diary Solution}

    Solution:
    \begin{minted}{python}
def main():
    date = input("Enter today's date (YYYY-MM-DD): ")
    entry = input("What's on your mind today? ")
    result = f"[{date}]: {entry}\n"
    with open("diary.txt", "a") as file:
        file.write(result)
    print("Diary entry saved successfully.")
main()
    \end{minted}
\end{frame}

\subsection{Exercise: Simple User Registration System}
\begin{frame}[fragile]
    \frametitle{Practice Time: Simple User Registration System}
    Background:
    Simulate a simple user registration system
    where new usernames are stored in a file.
    To prevent duplicate accounts, the program must check if a username
    already exists before adding a new one.

    Tasks: Write a Python program to manage a user list in file
    name \texttt{users.txt} (one username per line).
    \begin{enumerate}
        \small
        \item Prompt a new user to enter their desired username.
        \item Read the entire \texttt{users.txt} file to check if
              the new username already exists.
              Be careful to handle the newline characters
              \python{"\n"} properly.
              at the end of each line for accurate matching.
        \item If the username \textbf{already exists}, print the message:
              \python{"Username '[username]' already exists."}
              \python{"Please choose another one."}.
        \item If the username \textbf{does not exist}, append the new username
              to the end of \texttt{users.txt} and print a success message:
              \python{"Username '[username]' registered successfully."}
        \item \textbf{Exception Handling:} The first time the program runs,
              \texttt{users.txt} may not exist.
              The program should handle this case without error and
              proceed to register the new user (as any username will be unique).
    \end{enumerate}
\end{frame}
\begin{frame}[fragile]
    \frametitle{Practice Time: Simple User Registration System (Cont.)}
    Example Interaction:

    First Run (when \texttt{users.txt} does not exist):
    \begin{minted}{text}
Enter a username to register: admin
User 'admin' registered successfully.
    \end{minted}
    Second Run (registering an existing username):
    \begin{minted}{text}
Enter a username to register: admin
Username 'admin' already exists. Please choose another one.
    \end{minted}
    Third Run (registering a new username):
    \begin{minted}{text}
Enter a username to register: guest
User 'guest' registered successfully.
    \end{minted}
\end{frame}
\begin{frame}[fragile]
    \frametitle{Practice Time: Simple User Registration System Solution}

    Solution:
    \begin{minted}[fontsize=\small]{python}
def main():
    filename = "users.txt"
    username = input("Enter a username to register: ")
    try:
        with open(filename, 'r') as file:
            if username in (line.strip() for line in file):
                print(f"Username '{username}' already exists. "
                      f"Please choose another one.")
                return
    except FileNotFoundError:
        pass
    with open(filename, 'a') as file:
        file.write(username + '\n')
    print(f"User '{username}' registered successfully.")
main()
    \end{minted}
\end{frame}

\subsection{Exercise: Robust Grade Converter}
\begin{frame}[fragile]
    \frametitle{Practice Time: Robust Grade Converter}
    Background:
    We need a tool to convert a numerical score (0-100) into a
    letter grade (A, B, C, D, F).
    This tool must be robust enough
    to handle various invalid user inputs,
    such as text or out-of range numbers.

    Tasks: Write a Python program that:
    \begin{enumerate}
        \small
        \item Prompts the user once to enter a score between 0 and 100 in a loop.
        \item \textbf{Exception Handling:} Use a \python{try-except} block to
              handle the user's input. If the input is not a valid number,
              the program should catch the \python{ValueError}, and print
              \python{"Invalid input. Please enter a number."} and continue the loop.
        \item If the input is a valid number, check if it is within the range of 0 to 100.
              If the score is \textbf{out of range} (less than 0 or greater than 100),
              print \python{"Error: Score must be between 0 and 100."}.
        \item If the score is valid and in range, print the corresponding letter grade
              according to the rules below, and then end the program:
              90-100: A, 80-89: B, 70-79: C, 60-69: D, 0-59: F.
    \end{enumerate}
\end{frame}
\begin{frame}[fragile]
    \frametitle{Practice Time: Robust Grade Converter Solution}

    Solution:
    \begin{minted}[fontsize=\small]{python}
def main():
    while True:
        text = input("Enter a score (0-100): ")
        try:
            score = float(text)
            if score < 0 or score > 100:
                print("Error: Score must be between 0 and 100.")
                continue
            if score >= 90:
                grade = 'A'
            elif score >= 80:
                grade = 'B'
            pass # ... continue the grading logic
            print(f"Grade: {grade}")
            break
        except ValueError:
            print("Invalid input. Please enter a number.")
main()
    \end{minted}
\end{frame}

\subsection{Exercise: Palindrome}
\begin{frame}[fragile]
    \frametitle{Practice Time: Palindrome}
    Background:
    Write a program to read a positive integer number $n$.
    Use while-loop(s) to solve this exercise.
    If $n$ is a palindrome, print \python{"YES"}. Otherwise, print \python{"NO"}.

    Examples:

    \begin{center}
        \begin{tabular}{|c|c|}
            \hline
            Input        & Output       \\ \hline
            \texttt{1}   & \texttt{YES} \\ \hline
            \texttt{10}  & \texttt{NO}  \\ \hline
            \texttt{101} & \texttt{YES} \\ \hline
        \end{tabular}
    \end{center}
\end{frame}

\begin{frame}[fragile]
    \frametitle{Practice Time: Palindrome Solution}

    Solution:
    \begin{minted}{python}
def is_palindrome(number):
    original_number = str(number)
    reversed_number = ""
    while number > 0:
        digit = number % 10
        reversed_number += str(digit)
        number //= 10
    return original_number == reversed_number
def main():
    n = int(input())
    if is_palindrome(n):
        print("YES")
    else:
        print("NO")
main()
    \end{minted}
\end{frame}

\subsection{Exercise: Widest Fragment in a Sequence}
\begin{frame}[fragile]
    \frametitle{Practice Time: Widest Fragment in a Sequence}
    Write a program to read a sequence of integer numbers
    ending with \texttt{0}.

    Determine the length of the widest fragment
    (consecutive sub-sequence) where
    all the elements are equal to each other.

    Examples (newlines are replaced by spaces for clarity):
    \begin{center}
        \begin{tabular}{|l|c|}
            \hline
            Input                        & Output     \\ \hline
            \texttt{0}                   & \texttt{0} \\ \hline
            \texttt{1 2 3 4 5 0}         & \texttt{1} \\ \hline
            \texttt{3 3 3 3 0}           & \texttt{4} \\ \hline
            \texttt{1 1 2 2 2 2 3 3 3 0} & \texttt{4} \\ \hline
        \end{tabular}
    \end{center}
\end{frame}

\begin{frame}[fragile]
    \frametitle{Practice Time: Widest Fragment in a Sequence Solution}

    Solution:
    \begin{minted}[fontsize=\small]{python}
def main():
    max_length = 0
    current_length = 0
    previous_number = None
    while True:
        current_number = int(input())
        if current_number == 0:
            break
        if current_number == previous_number:
            current_length += 1
        else:
            current_length = 1
            previous_number = current_number
        if current_length > max_length:
            max_length = current_length
    print(max_length)
main()
    \end{minted}
\end{frame}

\section{Q\&A and Reminders}
\begin{frame}
    \frametitle{Any Questions?}

    Feel free to ask us about the tutorials and python coding!

    \begin{itemize}
        \item Contact us via email
        \item Ask us in person right now
        \item Post on Moodle or Ed
    \end{itemize}
\end{frame}

\begin{frame}
    \frametitle{Thank You!}

    Enjoy your coding journey in Python!

    \begin{itemize}
        \item Happy Coding and Good Luck in Your Final Exam!
        \item Submit your tutorial 10 exercises before \textbf{December 5 23:59}!
        \item Don't forget tutorial 8 and 9 exercises\\(due on \textbf{TOMORROW November 28 23:59})!
        \item Bonus: \LaTeX source code and compiled pdf of this slides available at
              \href{https://github.com/xtz206/COMP1117-Autumn25-Tutorials/releases}
              {https://github.com/xtz206/COMP1117-Autumn25-Tutorials/releases}
    \end{itemize}
\end{frame}

\end{document}